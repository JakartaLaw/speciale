\documentclass{article}
\usepackage[utf8]{inputenc}

%additional usepackages
\usepackage{amsmath}
\usepackage{eufrak}
\usepackage{tikz}
\usepackage{ amssymb }
\usepackage{import}
\usepackage[ruled,vlined]{algorithm2e}
\usepackage{booktabs}
\usepackage[capposition=top]{floatrow}
\usepackage{scrextend}
\usepackage{csquotes}
\usepackage{graphicx}
\usepackage{pdfpages}

\usepackage{caption}
\usepackage{subcaption}

% bibliography
\usepackage[backend=biber, citestyle=authoryear]{biblatex}
\addbibresource{speciale.bib}

% functions

\newcommand{\quickwordcount}[1]{%
  \immediate\write18{texcount -1 -sum -merge -q #1.tex output.bbl > #1-words.sum }%
  \input{#1-words.sum}%
}

\newcommand{\quickcharcount}[1]{%
  \immediate\write18{texcount -1 -sum -merge -char -q #1.tex output.bbl > #1-chars.sum }%
  \input{#1-chars.sum}(not including spaces)%
}


%%%% INDEPDENT SIGN %%%%

\makeatletter
% Taken from http://ctan.org/pkg/centernot
\newcommand*{\centernot}{%
  \mathpalette\@centernot
}
\def\@centernot#1#2{%
  \mathrel{%
    \rlap{%
      \settowidth\dimen@{$\m@th#1{#2}$}%
      \kern.5\dimen@
      \settowidth\dimen@{$\m@th#1=$}%
      \kern-.5\dimen@
      $\m@th#1\not$%
    }%
    {#2}%
  }%
}
\makeatother

\newcommand{\independent}{\perp\mkern-9.5mu\perp}
\newcommand{\notindependent}{\centernot{\independent}}

\newcommand{\E}{\mathbb{E}}
\newcommand{\N}{\mathbb{N}}
\newcommand{\ndist}{\mathcal{N}}
\newcommand{\std}{\mathbf{std}}
\newcommand{\R}{\mathbb{R}}
\newcommand{\lra}{\Leftrightarrow}

\newcommand{\Loss}{\mathcal{L}}
\newcommand{\loss}{\mathcal{l}}

\newcommand{\la}{\leftarrow}
\newcommand{\ra}{\rightarrow}

\newcommand{\ones}{\mathbf{1}}

\newcommand{\statespace}{\mathcal{S}}
\newcommand{\actionspace}{\mathcal{A}}

\DeclareMathOperator*{\argmax}{arg\,max}
\DeclareMathOperator*{\argmin}{arg\,min}

% paranthesis

\newcommand{\lp}{\left(}
\newcommand{\rp}{\right)}
\newcommand{\lsp}{\left[}
\newcommand{\rsp}{\right]}
\newcommand{\lcp}{\left\{}
\newcommand{\rcp}{\right\}}

% Margins and paragraph indent etc (layout)

\setlength{\parindent}{0em}
\setlength{\parskip}{0.25cm}
\renewcommand{\baselinestretch}{1.5}
\usepackage[margin=1.05in]{geometry}


% About Data
\title{\input{name}}
\author{Jeppe Søndergaard Johansen (pcv439)}
\date{May 2020}

\makeindex


\begin{document}


\includepdf[pages={1}]{frontpage/frontpage.pdf}


\maketitle

Number of characters: \quickcharcount{main}. Maximum is (144.000)

Number of spaces: \quickwordcount{main}

\begin{abstract}
This paper investigates reinforcement learning as a solution method for dynamic models. A discrete time, finite horizon, discrete choice model of female labour supply and fertility is formulated, and the model is solved using value function iteration and two reinforcement learning methods, namely deep Q-learning and double deep Q-learning. After estimating the model using method of simulated moments, and finding the simple model inadequate to describe data from Statistic Denmark, the model is extended. The extension consists of 14 + 1 states. The extended model is solved only using double deep Q-learning and estimated using simulated method of moments. The results of the extended model convincingly matches data from Statistics Denmark and contemporary findings by \textcite{kleven_children_2019}. I conclude that the field of economics should further investigate reinforcement learning, as it allows solving dynamic models with high dimensional state space.
\end{abstract}


\pagebreak

\tableofcontents

\pagebreak

\section{Hvad der skal nås:}
\begin{itemize}
    \item \textbf{(DONE)} Model \textit{kommentér på atrophy / human capital depreciation}
    \item \textbf{(DONE)} Litteraturteview 
    \item \textbf{(DONE 95 \%)} RL-teori. \textit{Kommenter på Bootstrapping som koncept of ret igennem. Skriv referencer}
    \item \textbf{DONE 75 \%)} Deep Learning - \textit{Der skal nok stå noget mere. Det skal også gøres skarpere. Skal lige have lidt afstand til det.}
    \item Løsninger
    \begin{itemize}
        \item \textbf{(DONE)}VFI (Deep Q-function Iteration) 
        \item \textbf{(DONE) }Q-learning \textit{mangler implementations detaljer + løsningsfigurer}
        \item \textbf{(DONE)}Double Q-learning \textit{mangler implementations detaljer + løsningsfigurer}
    \end{itemize}
    \item \textbf{(50 \% Done)} Parameter kalibrering!  \textit{Vær meget klar i spyttet om det er agent baseret modellering, for at komme hurtigere igennem}
    \item Estimation
        \begin{itemize}
        \item \textbf{(DONE)} VFI (Deep Q-function Iteration)
        \item \textbf{(DONE)} Q-learning
        \item \textbf{(DONE)} Double Q-learning
    \end{itemize}
    \item Resultater
\end{itemize}

\pagebreak

%\section{Introduction} 

This will be the introduction of this assignment.

\section{Review of Literature}\label{sec:lit_review}

\section{Model - Exogenous Fertility}

\begin{equation}
    \textbf{State space: }\statespace = \R^{2} \times \{0, 1, 2, 3, 4, 5\} \times \{ 18, 19, \cdots, 70\}
\end{equation}

\begin{equation}
    \textbf{Action space: }\actionspace  = 46 \cdot \{0, 15, 25, 37, 45\} 
\end{equation}

\begin{equation}
    \textbf{States: }\{G, Z, K, Q\}, \qquad \textbf{Actions: } \{H\} 
\end{equation}

The model represents the labour supply of married women - and the effect of children. More precisely it models a woman who can accumulate human capital by supplying a desired number of working hours. The woman can with a certain probability in each period give birth to a child, where, the number of children influences the utility. The model presented here has 4 states: $G$ which represents human capital, $Z$ which represents the agents idiosyncratic wage path, $K$ the number of kids owned by the agent and $Q$ which evolves in a deterministic fashion and represents the age of the agent. The action the agent can take in each period is a discrete number of working hours $H$.

\begin{equation}
    \textbf{Variables: }\{S, W, Y, M, U, L\},  \qquad \textbf{Parameters: } \{\beta_K, \beta_L, \beta_Y, p_\psi, \sigma_\epsilon, \delta, \zeta\}
\end{equation}


To elaborate the rest of the model variables: $U$ is the utility. Following the formulation of (Adda, Dustman, Stevens - The Career Cost of Children XXX), dividing the utility into sub-utility functions, where each sub-utility function allows for curvature by specifying a constant relative risk-averse (CRRA) function for each sub-utility. Assuming the special case of $\ln(\cdot)$. The parameters $\beta_K, \beta_L, \beta_YK$ is the individual weighing of the different sub-utilities. Note that for identification I will restrict $\beta_Y= 1$. $W_t$ the wage for given year as function of the supplied working hours and hourly salary $S$. The hourly Salary has three components: the minimum wage: $\zeta=120$, $G_t$ the human capital and at last $Z_t$ the idiosyncratic wage path. I assume that the three parts are additive. $Y_t$ represents the households total earning, where $M_t$ is the earnings of the husband, $W_t$ mentioned above is the salary of the woman. Lastly $M_t$, the husbands wage is assumed to be a deterministic function of age. It is modelled non parametric using data from Danmarks Statistik Bank.

The states of the agent evolves the following way: $Q_t$ evolves deterministcally. $K_t$ evolves by with a certain probability and ekstra children is added to the household. This probability is dependent on the age of the women $Q$. The idiosyncratic wage path of the women follows a random walk. The Human capital accumulates with a constant depreciation rate $\delta$. adding the number of working hours last period.

\begin{align}
    U_t(K_t, L_t, Y_t) &= \beta_K \ln(K_t) + \beta_L \ln(L_t) + \beta_Y \ln(Y_t) \\
    W_t(S_t, H_t) &=S_t \cdot H_t \\
    S_t(Z_t, G_t) &= \min(\zeta, (\zeta + Z_t + G_t))  \\
    Y_t(W_t, M_t) &= W_t + f^M(Q_t)\\
\end{align}


Law of Motion.

\begin{align}
    Q_{t+1}(Q_t) &= Q_t \\
    K_{t+1}(K_t, Q_t)  &= K_{t} + \psi_t, \qquad \psi_t \sim Bernoulli(p_\psi) \mid Q_t \\
    Z_{t+1}(Z_t) &= Z_t + \epsilon_t, \qquad \epsilon_t \sim \ndist(0, \sigma_\epsilon) \\
    G_{t+1}(G_t) &= G_t(1 - \delta) + H_t \\
\end{align}


\begin{figure}
    \centering
    \includegraphics[scale=0.1, angle=90]{figures/modeldynamic_tmp_exogenous.jpg}
    \caption{Model dynamics - Exogenous (TMP)}
    \label{fig:tmp_modeldynamics_exogenous}
\end{figure}

%\section{Reinforcement Learning}


\subsection{What is reinforcement learning}

Reinforcement learning is the study of "learning by interaction". As in the real world, when learning how to act, this is done by trial and error.

\subsection{The Environment/Agent interface}

The problem of learning by trial and error has a natural formulation by the environment/agent interface:

\begin{figure}
    \centering
    \includegraphics[scale=0.5]{figures/agent_environment_interface.png}
    \caption{Agent/Environment Interface}
    \label{fig:agent_enviroment_interface}
\end{figure}

The problem can be phrased the following way: An agent will over a series of discreet time steps $t=1, 2, \cdots, T$ take an action which will lead to some sort of reward (or in economic terms utility). So for each time step the agent gets information about the environments state $S_t$. The agent can then take an action $A_t$, which prompts the environment to return a reward $R_{t+1}$, and a new state $S_{t+1}$, which restarts the loop until the game terminates. The agents sole purpose is to maximize the cumulative rewards through out the game. It should be noted here that $R_t \in \R$. So the agent is optimizing over a sum of scalars. The representation of the state $S_t$ can be only partially observable in the reinforcment learning formulation. This game will lead to a trajectory of states, actions and rewards that look like:

\begin{equation}
    S_0, A_0, R_1, S_1, A_1, R_2, S_2, A_2, \cdots ,R_{T-1}, S_{T-1}, A_{T-1}, R_{T}
\end{equation}

However more formally certain assumption is necessary to perform any sort of modelling. The most fundamental assumption RL relies on is Markov Decision processes: A Markov decision process (MDP) can be described as:

\begin{equation}\label{eq:mdp1}
    p(s_t, r_t \mid s_{t-1},  a_{t-1}) = p(s_t, r_t \mid s_{t-1},\cdots, s_{0}, a_{t-1}, \cdots, a_{0}) = P(S_t = s_t, R_t = r_t \mid S_{t-1} = s_t, A_{t-1} = a_t)
\end{equation}

Breaking the equation \eqref{eq:mdp1} down we see that the MDP follows a true probability distribution. That is $R_t$ and $S_t$ is well defined probability distributions. More over another very important feature is that the probability distribution of $S_t$ and $R_t$ only depends on the last state and action. This turns out to be a very important assumptions to do any sort modelling. This implies that the state $s_t$ contains all relevant information about the past, which is what makes the problem a \textit{Markov} Decision Process. This condition/assumption yields certain important features. First, it implies that size of the probability distribution would not grow linearly as more and more states and actions was represented for the agent. Yielding the computations more and more expensive. Second thing which is important is, that allows for backward induction and dynamic programming (a topic which will be discussed later). Lastly this implies than when we model the system we must ensure that the state is represents all relevant information about the past.

From the equation \eqref{eq:mdp1} multiple statements can be derived, but most importantly one can derive, the expected rewards, which is what we is what the agent want to maximize:

\begin{equation}
    \E[R_t \mid A_{t-1} = a_{t-1}, S_{t-1} = {s_{t-1}}] = \int_{r_t} \int_{s_t} p(r_t, s_t \mid r_{t-1} s_{t-1}) d s_t d r_t 
\end{equation}

As mentioned in the start the agents goal is to maximize the cumulative rewards. This can be described as in equation:

\begin{equation}\label{eq:cum_rewards}
   G_t = R_{t+1}, R_{t+2}, R_{t+3}, \cdots R_{T}
\end{equation}

In general however the above formulation can be problematic with continuing tasks. That is if $T \rightarrow \infty$. Therefore discounting of rewards is usually implemented. Which have the nice economic implications, that agents in the real world tend to be impatient, and therefore more realistically real human agents:

\begin{equation}
    G_t = R_{t+1} + \gamma R_{t+2} + \gamma^2 R_{t+3} + \cdots = \sum_{k=0}^{T - t} \gamma^k R_{t+k+1}
\end{equation}

with $\gamma$ being the discount rate, yielding a geometric series, that is known to converge, if $R_k$ is bounded.

\subsection{Value function, Q-function, Policy function and the Bellman Equation}

To navigate in the environment usually a couple of things is considered: First the value-function:

\begin{equation}\label{eq:value_function1}
    v_{t}^{\pi}(s_t) = \E_t [G_t \mid S_t = s_t] = \E_t \lsp \sum_{k=0}^{T - t} \gamma^k R_{t+k+1} \bigg\vert S_t = s_t \rsp 
\end{equation}

Where it's implied in this formulation that the agent is following some policy $\pi$. A couple of things to note about \eqref{eq:value_function1}: Since the expectation is taken over a sum this is the same to take the sum over the expectations, yielding it possible to calculate the individual expected returns from following a policy, and using those to calculate the value function. Another formulation that lies close to the value-function is the Q-function which will be explored in this paper:

\begin{equation}
    q_t^{\pi} (s_t, a_t) = \E_t [G_t \mid S_t = s_t, A_t = a_t ] = \E_t \lsp \sum_{k=0}^{T - t} \gamma^k R_{t+k+1} \bigg\vert S_t = s_t, A_t = a_t\rsp 
\end{equation}

which only differ from the value function by also conditioning on the action, and not only the state. The value function and the Q-function shares the property that it maps the expected value of an state (or state action pair) to a scalar vale, where this value represent the cumulative, discounted rewards of following a certain policy. This has the nice property of allowing the agent to choose an action which maps to the highest expected value.

A policy function is a how the agent chooses its actions:

\begin{equation}
    \pi_t : \statespace \mapsto \actionspace 
\end{equation}

Some of the methods explored in this paper works by directly estimating the policy function, where as other works by estimating the value function or Q-function using these to find the correct optimal policy.

Using the expression of the value function we can now express the Bellman equation:

\begin{equation}
    v^{\pi}_{t} (s_t) = \E_t [G_t \mid S_t] = \E_t  \lsp R_{t+1} + \gamma G_{t+1} \mid S_t \rsp = \E_t \lsp R_{t+1} + \gamma v_{t+1}^{\pi}(s_{t+1}) \rsp
\end{equation}

A couple of things to note about the bellman equation: First and foremost one should consider that $\E [v_{t+1}^\pi(s_{t+1})]$ does not imply $v_{t+1}^{\pi}(\E[s_{t+1}])$, which has the consequence of considerable computational markup when solving a model using the Bellman Equation. Furthermore.

\subsection{Relationship to Dynamic Programming}\label{sec:dynamic_programming}

\textbf{BOOTSTRAP SKAL INTRODUCERES}

Dynamic programming, invented by Richard Bellman, and allows for a way to find the optimal policy $\pi^{*}$ and the associated value function $v^{\pi^{*}}$ when following the optimal policy. To do dynamic programming, certain things are necessary. First the size of the statespace should be limited. That is due to the fact, that number of computations will increase exponentially with the number of states. Secondly it also requires that the entire MDP is known. So partially observable models f.x. cannot be used. In general it can be said that dynamic programming (and also reinforcement learning) is to use the value function to structure the search for good policies  (XXX Sutton and barto). It's clear that if we have the optimal value function, we also have the optimal policy:

\begin{equation}
    v_t^{*}(s_t) = \underset{a_t}{\max} \lsp \E \lsp R_{t+1} + \gamma v^{*}_{t+1}(s_{t+1}) \mid S_t = s_t, A_t = a_t\rsp \rsp
\end{equation}

Dynamic programming problems can be solved by two different approaches: value function iteration and policy function iteration.

Policy function iteration consists of two steps: 1) an evaluation step. Which calculates the value of a policy, and 2) a policy improvement step.
Step 1 can be considered a prediction step, that by following a given policy calculates the associated value function for said policy. This is done by sweeping through the state space calculating the expected value of the state following the policy. This process continues until the algorithm has converged, where converged implies that the difference between the previous estimation of the value function and the current estimation of the value function, only differs below some threshold. The second step, policy improvement, follows the logic. Assume that the policy just used for the evaluation step is optimal. Then there should be no other strategy that would yield a higher value-function for all possible states. Now this can be written more formally as:

\begin{equation}
    \pi^*_t (s_t) \geq \tilde{\pi}(s_t)\qquad \forall s_t \in \statespace
\end{equation}

where $\tilde{\pi}$ is any arbitrary strategy. This also implies that if we at any point in the state space can find a policy that yields a higher value function than the current, then we should switch strategy. Which have now yielded a better strategy! If the during the sweep through state space, searching for better policies, a new policy is found. The process goes back to a policy evaluation step. This goes on until that there can be found no better policy than currently being employed. A graphical the process can be considered as shown in equation \eqref{eq:policyevaluation} where $\overset{\textbf{E}}{\longrightarrow}$ denotes a policy evaluation and $\overset{\textbf{I}}{\longrightarrow}$ denotes policy improvement:

\begin{equation}
    \label{eq:policyevaluation}
    \pi^0 \overset{\textbf{E}}{\longrightarrow}
    v^{\pi^0} \overset{\textbf{I}}{\longrightarrow} \pi^1 \overset{\textbf{E}}{\longrightarrow} v^{\pi^1} \overset{\textbf{I}}{\longrightarrow} \cdots \overset{\textbf{I}}{\longrightarrow} \pi^* \overset{\textbf{E}}{\longrightarrow} v^{\pi^*}
\end{equation}

The second approach value function computes a max over the value function implying only a single sweep through the state space in iteration of the loop, yielding it a much faster approach for solving the model. Value function iteration can be considered using the Bellman equation as an update rule. (Barto and Sutton XXX). The algorithm for value function iteration is described below:

\begin{algorithm}[H]
\SetAlgoLined
\KwResult{Yielding $\pi^*, v^*$}
 Algorithm parameter $\theta > 0$ determining accuracy of estimation\;
 Initialize $V(s)\quad \forall s \in \statespace$ except $V(terminal) = 0$\;
 \While{$\Delta > \theta$}{
    $\Delta \la 0$ \; 
    \ForEach{$s \in \statespace$}{
        $v \la V(s)$ \;
        $V(s) \la \underset{a}{\max} \E [R_{t+1}  + \gamma V(s') \mid A_t = a, S_t = s] $ \;
        $\Delta \la \max (\Delta, \mid  v - V(s) \mid )$
    }
 }
 \caption{Value Function Iteration}
\end{algorithm}

 So for each step sweep through the loop a single sweep through the state space is performed finding which action yields the highest associated value for a given state. Finally when the sweep is completed the associated policy evaluation can be done. Finding the value function. Again this algorithm terminates when the difference between the value function of the last sweep and the current value function is below some threshold.
 
 In economics very often a dynamic model will be solved using value function iteration but with the addition of using backwards induction. How is this possible. Just as the model presented in this paper. The model assumes an agent acting over $T$ time steps, terminating when the agent reaches a certain age. So in a sense the agent moves in a deterministic fashion towards the termination of the environment. This implies that one can solve such a model by only doing a single sweep through the state space! What will be done is that at the terminal period the agent will optimize his value function. Now using this value associated with the terminating period, the agent can consider his actions in $T-1$. Remembering that we can use the Bellman equation as an update step we can write:
 
 \begin{equation}
     V_{T-1}(s_{T-1}) = \underset{a_{T-1}}{\max}\E [R_{t+1} + \gamma V_T (s_T)]
 \end{equation}
 
 In other words, we can model all possible states that the agent can encounter in each time step of the model, making it possible to find the optimal value function $v^{\pi^*}$ and policy function $\pi^{*}$ by simple sweep through the state space using a combination of dynamic programming and backwards induction. The implementation of Value function iteration in this paper will be explored in a later section.
 
 \subsection{Overview of other Reinforcement learning techniques}.
 
Later in this paper I present three different reinforcement learning methods. Here i present some basic information that makes the reader able to digest the material presented. First it's important to address why not only use dynamic programming. Dynamic programming requires that a perfect model of the environment is accessible. This is due to the fact, that when calculating the expected value function for each action, the probability distribution of the reward and the next state, needs to formulated in an explicit form. The other techniques presented here does not have the same requirement. Secondly DP methods require that state space cannot be to large. The implementations presented later does not have the same requirements, allowing for approximating the value function and/or the policy function. Below i explain two different methods of learning \textit{Monte Carlo Methods} and \textit{Temporal Difference Learning} these being the two foundations of the reinforcement learning methods presented later.

\subsubsection{Monte Carlo Methods}

Before delving into MC-methods it is appropriate to introduce some more nomenclature: when talking about an \textit{episode} it should be understood as an agent moving through the environment from start to termination. When talking about a \textit{step} it should be understood as going from one state to the next in the environment. The best way to get a sense of Monte Carlo methods is to present the algorithm:

\begin{algorithm}[H]
\SetAlgoLined
\KwResult{Yielding $v^{\pi}$}
 Input: policy $\pi$ to be evaluated\;
 $Returns(s) \la$ an empty list $s \in \statespace$\; 
 \While{Forever}{
    Generate an episode: $S_0, A_0, R_1, S_1, A_1, \cdots S_{T-1}, A_{t-1}, R_T$\; 
    $G \la 0$ \;
    \ForEach{step in episode, $t = \{T -1 , T-2 , \cdots, 0\}$}{
        $G \la \gamma G + R_{t+1}$\;
        \If{$S_t \notin \{S_{t-1}, S_{t-2}, \cdots S_0 \}$}{
            Append $G$ to $Returns(S_t)$ \;
            $V(S_t) \la average(Returns(S_t))$ \;
        }
    }
 }
 \caption{First-Visit MC prediction, for estimating $v^{\pi}$}
 \label{alg:mcfirstvisit}
\end{algorithm}

Algorithm \ref{alg:mcfirstvisit} shows how the general concept of estimating the value function for given policy. Things to be noted: First and foremost the algorithm presented above is made to a tabular case, i.e discrete state space. This will be dealt with in the algorithm presented later in this paper. Second it is assumed that the policy stays constant. That is, the policy does not update as more and more episodes are experience, which defeats the purpose of learning how to interact with the environment. Addressing the latter part usually the policy is updated, and the you could discard the old experience. However another possibility is to accept the non stationarity of the data collected, and update the policy using the data from a previous policy, hoping that with time the algorithm will converge.

Consider now how to do Monte Carlo Control, i.e. approximating the optimal policy. Just as with policy iteration the pattern followed is:

\begin{equation}
    \label{eq:montecarlocontrol}
    \pi^0 \overset{\textbf{E}}{\longrightarrow}
    q^{\pi^0} \overset{\textbf{I}}{\longrightarrow} \pi^1 \overset{\textbf{E}}{\longrightarrow} q^{\pi^1} \overset{\textbf{I}}{\longrightarrow} \cdots \overset{\textbf{I}}{\longrightarrow} \pi^* \overset{\textbf{E}}{\longrightarrow} q^{\pi^*}
\end{equation}

Just with the caveat that instead of considering the value function, the Q-function (state-action pair) is used for policy evaluation. Policy improvement is done by making the policy greedy with respect to the current estimated Q-function (XXX Barto and sutton). Since the Q-function instead of the value function is used, no model us needed to construct a greedy policy (XXX barto and Sutton). The greedy policy is the one that for each $s \in \statespace$ deterministically chooses an action with maximal action-value:

\begin{equation}
    \pi(s) = \underset{a}{\argmax}  q( s, a )
\end{equation}

Now this algorithm is made under the assumption of \textit{exploring starts} and under the assumption of \textit{infinite epsiodes}. The assumption of inifinte episodes is to ensure convergence, and in practice the algorithm is usually run until the algorithm has converged by some high number of episodes. The more problematic assumption is that of exploring starts. This is due to the fact, that in reality we would not assume that there is a uniform distribution of starting in all states $s \in \statespace$, and proceed the episode from that starting point. This is an essential assumption, because otherwise, one could not know the value function of unexplored states without visiting them. This leads to the concept of on-policy control. On policy control is most often implemented using an $\epsilon$-greedy strategy. The implication is that $P(A=a \mid S=s, q^{\pi}) > 0, \forall a \in \actionspace, s \in \statespace$. Where the non optimal action is choosen by drawing from an uniform distribution over all actions with probability $\epsilon$. Off-policy control works by having two different policies one which works by doing the optimal action (target policy $\pi$), and a second policy that moves to investigate the state space (behavior policy $b$). Using the experience from these two policies importance sampling can be used to estimate the value function! It's required that the behavior policy has a non zero probability to any action for any state.

\subsubsection{Temporal Difference Learning}

Temporal Difference (TD) learning is a combination of ideas taken from Dynamic Programming and Monte Carlo Methods. It takes from Monte Carlo methods, that you do not need a perfect model of the environment. It takes from Dynamic Programming the bootstrapping, i.e. it uses learned estimates to update, without needing the final outcome. Just as Monte Carlo methods, Temporal difference methods has a prediction and control element. 

TD prediction can be summarized in the equation:

\begin{equation}
    V(S_t) \leftarrow V(S_t) + \alpha \lsp R_{t+1} + \gamma V(S_{t+1}) - V(S_t) \rsp
\end{equation}

The equation states that $V(S_t)$ should be updated according to the return in a given period + the value function of the next period $V(S_{t+1})$. This method is called TD(0), since it only uses 1 step to update the value function. In essence TD methods learn a guess from a guess. This allows for not having a complete model of the environment (of rewards and next-state probability distributions). Compared to MC methods the TD learning can learn at each step - update it's estimate at each point in time. It is also proven, that for any policy $\pi$ kept fixed, then a TD(0) algorithm will converge to $v^{\pi}$ (XXX Barto and Sutton). Even though, an open mathematical question, in practice it is found that TD methods converge faster than MC methods (XXX Barto and Sutton).

Control with temporal difference learning, can also be separated into on-policy control and off-policy control. The most used on-policy control for TD methods is called SARSA (state, action, reward, state, action). Just as with MC methods a need to trade off exploitation and exploration is present. The agent want to see new parts of the state space (exploration), but should also use exploit what is assumed to be the optimal choice given the value function associated with the current value function. In this case instead of using the value function instead, consider the case of using the state-action pair function $Q(S_t, A_t)$, yielding the update rule:

\begin{equation}
    Q(S_t, A_t) \leftarrow Q(S_t, A_t) + \alpha \lsp R_{t+1} + \gamma Q(S_{t+1}, A_{t+1}) - Q(S_t, A_t) \rsp
\end{equation}

This update is made after each step in a non-terminal state of the environment. As in on-policy methods $q^{\pi}$ is estimated for the policy $\pi$ whilst at the same time greedy updating $\pi$ with respect to $q^{pi}$. Exploration can again be done using an epsilon greedy approach.

Off-policy control can be done by using Q-learning. Q-learning has the slight twist on update the rule for SARSA that:

\begin{equation}
    Q(S_t, A_t) \leftarrow Q(S_t, A_t) + \alpha \lsp R_{t+1} + \gamma \underset{a}{\max} Q(S_{t + 1}, a) - Q(S_t, A-t) \rsp 
\end{equation}

Actions is still chosen by an $\epsilon$-greedy policy, the difference here being the updating rule presenting a different policy than the actual policy, since we assume that we greedily chooses the action in time $t+1$ conditional on the state. Q-learning and double Q-learning will be explored in depth later when describing the algorithms used in this paper.

\subsection{Landmarks}

Finally, before moving a brief overview of environments/games where reinforcement learning have been instrumental. The first big success of reinforcement learning was made by Tesauro in 1992 creating an agent of learning to play backgammon trough self play. Using an artificial neural network to approximate the value function, and using a temporal difference algorithm, the algorithm was capable of playing expert level backgammon. In 2011 the IBM Watson algorithm won in jeopardy using the same methods as Tesauro for his backgammon agent. In 2013 the company DeepMind (now acquired by google), showed that it was possible for a reinforcement learning algorithm to learn to play video games. Here an important feat was, that it was fed the raw image input and used an artificial neural network to transform this image into an representation of the state space allowing for learning how to navigate in the environment. in 2016 DeepMind created AlphaGo and a year later AlphaGo Zero, which learned to master the game of Go. This was assumed in a long time to be a hard problem for learning algorithm due to its very large state and action space. The first iteration used expert players to learn the game, while the AlphaGo zero used only selfplay. These examples show that the feasibility of these algorithms to learn to navigate in complicated environments, might leave a way for high dimensional dynamic economic models to be solved using reinforcement learning methods.

%\section{Deep Learning}

Deep learning is a subset of machine learning or sometimes referred to as statistical learning. Machine learning essentially is just statistics. However usually in statistics a model is explicitly formulated. A typical example could be linear regression. The covariates is decided on, an tweaked such that they fit the model. The linear regression will yield a parameter vector which can then be interpreted. And this interpretation is usually the focus - parameter inference. Example: Does an increase in the minimum salary have a negative effect on BNP pr. capita. Machine Learning focuses on prediction. That is, we want to predict some target conditional on some covariates. And the specific model is not necessarily important. Instead a focus on Out-Of-Sample error is the focus. Deep learning which is used in the algorithm presented in this paper is in fact just layered non linear functions. These functions could f.x.  be sigmoid, tanh or rectified linear units. Using the chain-rule the deep learning algorithm can be used for numerical optimization using a gradient descent algorithm. The Objective function in this case is the loss-function i.e. the function that measures how well the algorithm perform.

\begin{equation}
    \frac{\partial}{\partial\theta} \Loss = \frac{\partial}{\partial\theta} \lp \sum \ell_i \rp = \frac{\partial}{\partial\theta} \lp \sum \lp \hat{Y}_i  - Y_i \rp^2 \rp = \frac{\partial}{\partial\theta} \lp \sum \lp f^{\theta}(X_i)  - Y_i \rp^2 \rp
\end{equation}

\subsection{Historical overview}

\subsection{Theoretical considerations}






\section{Tuning of parameters}

løn parametre

Jeg har brugt lons20 til at finde en distribution af lønnen, på tværs af alder.
Jeg har brugt lons50 til at finde trenden for lønniveauet betinget på alderen.

Jeg vil bruge dette data til at tune parametrene som styrer løn-niveau processen i modellen.

\iffalse
\begin{table}[ht]
    \centering
    \begin{tabular}{lrrrr}
\toprule
{} &    mean &    std &  skew &  kurtosis \\
gender &         &        &       &           \\
\midrule
male   &  242.82 &  53.51 &  1.81 &      8.33 \\
female &  213.31 &  33.38 &  0.56 &      1.34 \\
\bottomrule
\end{tabular}

    \caption{Wage distribution moments}
    \label{tab:my_label}
\end{table}


\begin{table}[ht]
    \centering
    \begin{tabular}{lrr}
\toprule
{} &  total obs count &  nan count \\
gender &                  &            \\
\midrule
male   &              601 &        306 \\
female &              601 &        367 \\
\bottomrule
\end{tabular}

    \caption{Wage distribution summary}
    \label{tab:my_label}
\end{table}

\begin{figure}
    \centering
    \includegraphics[scale=0.4]{figures/wage_distribution_lons20.png}
    \caption{Wage Distribution Men \& Women}
    \label{fig:my_label}
\end{figure}


\begin{figure}
    \centering
    \includegraphics[scale=0.4]{figures/wage_trend_lons50.png}
    \caption{Wage Trend Men \& Women}
    \label{fig:my_label}
\end{figure}

\begin{figure}
    \centering
    \includegraphics[scale=0.4]{figures/men_vs_women_hours_empirical.png}
    \caption{Working hours by age group (Men \& Women)}
    \label{fig:women_vs_men_hours_emmpirical}
\end{figure}

\begin{figure}
    \centering
    \includegraphics[scale=0.4]{figures/fertility_women_empirical.png}
    \caption{Fertility of women}
    \label{fig:fertility_by_women}
\end{figure}

\begin{figure}
    \centering
    \includegraphics[scale=0.4]{figures/wealth_by_age_group_empirical.png}
    \caption{Wealth by age group}
    \label{fig:wealth_by_age_group_empirical}
\end{figure}
\fi

\subsection{Tuning the Wage process}

To solve the model i will need some initial reasonable numbers for the parameters of the wage process. The wage process contains the following parameters: $(\eta_G, \eta_{G^2}, \delta, \alpha, \sigma_{\epsilon})$. As mentioned in the model specification in section \ref{sec:model1} i specify a wage process that follow the Mincer earnings equation, where i do not account for education (this is not part of the state space), and where the idiosyncratic wage path is added linearly to the wage. I assume that the parameters driving the wage process are the same for both men and women, and these can be by tuned independent of the entire model. This is primarily due to computational constraint, since tuning multiple parameters in the same time suffers from the \textit{curse of dimensionality}. Another reason is that the wage of the husband in the model is assumed perfectly deterministic, which imply I am not able to tune the parameters driving the wage path under the assumption these are the same for men and women. As mentioned in the model specification allowing for linearly added idiosyncratic wage path that follows a random walk, allows for tuning $(\eta_G, \eta_{G^2}, \delta, \alpha)$ that drives the age and sex specific expected wage level (referred to as \textit{wage path}), while tuning the variance at each point in time can be tuned by $\sigma_\epsilon$ (referred to as \textit{wage variance}). In other words the parameter tuning of the wage process is broken into two phases: First tuning age and sex specific expected wage level, second tuning the variance of the wages.

I tune the wage path by using a \textbf{LONS50}, a data set from Statistics Denmark (Danmarks Statistik Bank). This data set contains the wage trend for men and women at any given age. My objective is to minimize the squared difference between this wage path and a predicted wage path by simulating from the partial model with given parameters and taking the average wage path. Certain things should be noted about this partial model. The wage path is a function of human capital which is the choice variable of the model specified in section \ref{sec:model1}. This obviously makes it problematic to tune the parameters. I work around the problem by using the data set \textbf{LIGEF15} containing the number of worked hours for both men and women, which is supplied by Statistics Denmark. Since these numbers do not take into account people leaving the labour force temporarily - which we know women to do when giving birth. I model this the following way. I assume that women leave the work force for 1 year when giving birth to a child. I use the data set \textbf{FOD33} to get fertility rates of women. Again this data is supplied by Statistics Denmark. Formally this can summed up to:

\begin{equation}
    H_t = \begin{cases}
        H^{men}(Q) & \text{if sex=\textit{male}} \\
        H^{women}(Q) & \text{if sex=\textit{female} and birth=\textit{false}} \\
        0 & \text{if sex=\textit{female} and birth=\textit{true} }\\
    \end{cases}    
\end{equation}

The rest of the wage process follows the model described in section \ref{sec:model1}. The minimization problem follows the following process:

\begin{equation}
    \underset{\eta_G, \eta_{G^2}, \delta, \alpha}{\argmin} = \frac{1}{2}\frac{1}{(60 - 18)} \lp \sum_{q=18}^{60} \lp w^{men}_{q} - \frac{1}{500}\sum_{n=0}^{500}\tilde{w}^{men}_{n,q} \rp ^2 + \sum_{q=18}^{60} \lp w^{women}_{q} - \frac{1}{500}\sum_{n=0}^{500}\tilde{w}^{women}_{n,q} \rp ^2 \rp
\end{equation}

where $\tilde{w}_{n, q}$ denotes a simulated wage at age $q$ for the $n'tn$ simulated individual and $w_q$ is the true value wage level for a given age. I solve the minimization problem using Nelder-Mead (Simplex method). Essentially The simplex method allows for numerical optimization without the need to supply neither the gradient, Hessian or Jacobian matrix. I initialize the algorithm with the values: $(\alpha=4, \eta_G = 0.15, \eta_{G^2}=0.1, \delta=0.7)$, and let the algorithm run for a maximum of 100 iterations. 

%\section{Soution methods}

\subsection{Value Function Iteration}

As a benchmark solution, I solve the problem by using value function iteration. However modifications to the algorithm presented in section \ref{sec:dynamic_programming}. First and foremost, i consider the Q-function instead of the value function, allowing to store state-action value pairs. Second it should be noted that in this formulation multiple of the features are continuous, not allowing for tabular solutions. Thirdly, the dimension of the state-space + the size of the grid made in infeasable\footnote{In my initial attempt to solve the model by value function iteration, i attempted to get the expectation of the value function using Gauss-Hermite integration, and discretizing the statespace. The solution time was infeasible, which was why my approach changed.} to solve the model by classical ways of solving a model using backwards induction an discreetizing the state space. 

The model is solved using backwards induction. This is due to the fact the model terminates in a deterministic fashion when the agents reaches a certain age. For each step a large random sample of states is drawn conditional on a given age. For each of these state the agent takes each of the possible actions storing the results. This way a large sample of rewards and states can be generated. Furthermore for each action taken (if the state is not terminal) the the Q-function can be evaluated in the new state, taken the max of each possible action, allowing for the estimation of the value function. A graphical representation of the concept is shown in figure \ref{fig:vfi_figure}. 

\begin{figure}[ht]
    \centering
    \includegraphics[scale=0.15]{figures/vfi_figure.png}
    \caption{Value function iteration using Q-function}
    \label{fig:vfi_figure}
\end{figure}

The Q-function for each action in the actionspace is made by mapping each $a\in \actionspace$:

\begin{equation}
 Q(S, A=a) = R_t + \gamma \underset{a'\in \actionspace}{\max} Q(S', a')
\end{equation}

In that sense this can be consideread akin to monte carlo integration, however instead of appoximating the expectation in a single point, rather find the distribution of the rewards + discounted value function over the entire state space. The idea is to approximate the integral by using a statistical method, and in this case Deep Learning (another machine learning method would be equally good). Consider $f$ to be a machine learning function, that has the property:

\begin{equation}
    f: \statespace \mapsto\R^{\mid \actionspace \mid}
\end{equation}

For a given point in state space a prediction of the value function is computed for each possible action. This implies the method only is feasible for discrete state space. By trying to reduce the mean squared error between the the true values of the Q-function, and the prediction, the $\E[Q(a, s)]$ can be found, which corresponds to integration as could be done using f.x. Gauss Hermite or Monte Carlo integration.

\begin{algorithm}[H]
\SetAlgoLined
\KwResult{Write here the result }
 Initialize $\tilde{Age} = Age_{max}$\;
 Initialize empty lists for storing results: $X, Y$\;
 Initialize memory counter $j=1$\;
 \While{$\tilde{Age} > Age_{min}$}{
  Draw $\{s_i\}_{i=1}^{i=N}$, where $s_i \sim \statespace \mid Age=\tilde{Age}$ \;
  \ForEach{$s_i$}{
  Create empty array $Z$ of length $\mid \actionspace \mid$\;
  \eIf{$\tilde{Age}= Age_{max}$}{
   \ForEach{$a_k \in \actionspace$}{
    $Z[k] \leftarrow r(s_i, a_k)$ \;
   }
   }{
   \ForEach{$a_k \in \actionspace$}{
    $Z[k] \leftarrow r(s_i, a_k)$ + $\gamma \underset{a' \in \actionspace}{\max}\lsp \hat{q}(a', s_i') \rsp$\;
    }
  }
  $Y[j] \leftarrow Z, X[j] \leftarrow s_i$\;
  $j = j + 1$\;
  }
  Estimate $\hat{Q}$ by training a Deep NN using samples from $X, Y$.
 }
 \caption{Deep Q-function iteration solution method}
 \end{algorithm}

Since i use a deep neural network to approximate the $Q$-function, certain things need to be considered. First and foremost I need to consider the architecture of the network. Next I need to consider the train

For each age i draw 20.000 random samples. This is because any smaller number of draw seemed to be detrimental to the performance. This is inline with standard Deep Learning practices. These kinds of network is known to be very data hungry. When training the network i draw a random sample of 100.000 observations. If I have not yet accumulated 100.000 observations the algorithm draws all observations. The architecture of network is fairly simple being a two-layer fully connected network. First layer being 16 nodes wide, second fully connected layer being 8 nodes wide. I found that mini batching, did not seem to work well on this particular task, and instead i train on all observations, using a validation split of 30 \%, training for a maximum of 150 epochs and finally i allow for early stopping, that is, when the validation loss is not furthering decreasing i stop the training of the network. I do allow the algorithm a patience of 5. Implying that the algorithm will try to lower it's validation loss for five additional epochs before terminating the training.


\begin{figure}[ht]
\begin{subfigure}{.5\textwidth}
  \centering
  \includegraphics[width=1\linewidth]{figures/dqi_model1_beta_2_solution_benchmark_paths.png}
  \caption{Simulated Paths}
  \label{fig:dqi_solution_beta2_path}
\end{subfigure}%
\begin{subfigure}{.5\textwidth}
  \centering
  \includegraphics[width=1\linewidth]{figures/dqi_model1_beta_2_solution_benchmark_variance.png}
  \caption{Variance of Paths}
  \label{fig:dqi_solution_beta2_var}
\end{subfigure}
    \caption{Value Function Iteration solution vs. benchmark $(\beta_L = 2)$}
    \label{fig:dqi_solution_beta2}
\end{figure}

\begin{figure}[ht]
\begin{subfigure}{.5\textwidth}
  \centering
  \includegraphics[width=1\linewidth]{figures/dqi_model1_beta_4_solution_benchmark_paths.png}
  \caption{Simulated Paths}
  \label{fig:dqi_solution_beta4_path}
\end{subfigure}%
\begin{subfigure}{.5\textwidth}
  \centering
  \includegraphics[width=1\linewidth]{figures/dqi_model1_beta_4_solution_benchmark_variance.png}
  \caption{Variance of Paths}
  \label{fig:dqi_solution_beta4_var}
\end{subfigure}
    \caption{Value Function Iteration solution vs. benchmark $(\beta_L = 4)$}
    \label{fig:dqi_solution_beta4}
\end{figure}

Figure \ref{fig:dqi_solution_beta2} and \ref{fig:dqi_solution_beta4} shows the results of the Value Function Iteration algorithm\footnote{The plots use the name DQIteration (Deep Q-Function Iteration) instead of Value Function Iteration.} compared to 4 benchmarks. Three deterministic agents working either $0$, $37$ or $45$ hours pr. week and one agent taking random actions. Figure \ref{fig:dqi_solution_beta2} shows the utility for each step over the life cycle when the preference for leisure is fixed at $\beta_L = 2$. As the figure shows the DQ Agent (VFI agent) learns to navigate the environment, just as well as the best deterministic agent. Looking to the right hand side plot of \ref{fig:dqi_solution_beta2} it's clear that there is a great overlap between the two agents. The variance of the path is represented as one standard deviation of the utility for a given age for all episodes of the given agent. Figure \ref{fig:dqi_solution_beta4} compares the benchmark agents with the VFI solution when considering a preference for leisure $\beta_L = 4$. Again the VFI solution is as good as the best benchmark.

\newpage

\subsection{Deep Q-learning}

In this paper I implement the algorithm used by \parencite{mnih_playing_nodate} for beating Atari games as a way to solve model. First note here, however few modifications will be made to the original implementation, due to the fact, the environment they were only using sensory data (an RGB representation of an image), where a part of their achievement was to transform these images into features that which the value function accurately could map into scores of the game. 

Just as described in section \ref{sec:rl_theory} the algorithm tries to maximize the Bellman equation. However now the value-function is estiamted using Deep neural network as a function approximator. Mathematically this can be described as finding:

\begin{equation}
    Q^*(s_t,a_t) = \E [r_t + \underset{a_{t+1}}{\max}  Q^*(s_{t+1}, a_{t+1}) \mid s_t, a_t]
\end{equation}

However here $Q^*(s, a)$ is approximated by a parametric function (in this case a Deep Neural Network) $Q(s, a ; \theta)$. Folloing the terminology of \parencite{mnih_playing_nodate}, this function approximator is referred to as the Q-network. This Q-network can be trained using stochastic gradient descent as described in section \ref{sec:deep_learning}:

\begin{equation}
    \Loss_i(\theta_i) = \E \lsp (y_i - Q(s_t, a_t ; \theta_i))^2 \rsp
\end{equation}

where:

\begin{equation}
    y_i = \E [r_t + \gamma \underset{a_{t+1}}\max Q(s_{t+1}, a_{t+1}; \theta_{i-1}) \mid s_t, a_t ]
\end{equation}

where $i$ implies the iteration of the algorithm, such $i$ increments by one for each update of the parameters $\theta$. This implies the update equation determining the update:

\begin{equation}
    \nabla_{\theta_i} \Loss_i (\theta_i)  \E \lsp \lp y_i - Q(s_t, a_t ; \theta) \rp \nabla_{\theta_i} Q(s_t, a_t; \theta_i) \rsp
\end{equation}

Where i follow the formulation of \parencite{mnih_playing_nodate} that $Q(s_{t+1}, a_{t+1}; \theta_{i-1})$ is held fixed, allowing for just writing $y_i$, and not the parametric form of $y_i$.

A traditional choice would be to update the weight after each step in the algorithm only using the last sample. This would the correspond to the traditional Q-learning algorithm. This was the approach used in by to create the TD Gammon created by Tesauro\footnote{I have not been able to get acces to the original paper, so I have relied on the description made by Sutton and Barto.} as described by \parencite{sutton_reinforcement_2018}. However, this approach of using Q-learning with a non-linear function approximator has been shown to diverge, and did not extend itself well to learning any other game than backgammon \parencite{tsitsiklis_analysis_1997}. To accommodate this problem a replay buffer is implemented. At each step in a entry is made to the replay memory containing $s_t, a_t, r_t, s_{t+1}$. This data set $\mathcal{D}$ has a capacity of $N$ entries. Using this replay memory a random sample (mini batch) is drawn used to update the weights of the Q-network. Note here that the capacity of the memory buffer should be greater, by a substantial margin than the number of samples drawn. The random sample has a couple of advantages. It has decorelated observations to update the weights on, allowing for better training \parencite{mnih_playing_nodate}. Using a replay buffer requires off-policy learning which is the reason for using Q-learning. This is due to the fact that current parameters $\theta$ is not the same as those generating the data. Note that experiences is drawn randomly, and no experiences (which could have important insights) is prioritized. The full algorithm is summarized below in algorithm \ref{alg:dqlearning}.

\begin{algorithm}[H]
\SetAlgoLined
 Initialize replay memory $\mathcal{D}$ with capacity $N$\;
 Initialize action-value function $Q$ with random weights\;
 Initialize memory counter $j=1$\;
 \ForEach {episode $\in \{1, 2, \cdots M \}$}{
  Initialize sequence with an initial state $s_1$. This is drawn randomly.\;
    \For{$t \in \{1, 2, \cdots, T \}$}{
    With probability $\epsilon$ select a random action $a_t$\;
    Otherwise select $a_t = \underset{a_t}{\argmax}Q^*(s_t, a_t ; \theta)$\;
    Execute action $a_t$ in environment and observe reward $r_t$ and the new state $s_{t+1}$\;
    Store transition $(s_t, a_t, r_{t+1}, s_{t+1})$ in $\mathcal{D}$\;
    Sample random mini-batch of transitions $(s_t, a_t, r_{t+1}, s_{t+1})$ from $\mathcal{D}$\;
    Set $
      y_j = \begin{cases}
        r_j & \text{if terminal states} \\
        r_j + \gamma \underset{a_{t+1}}{\max}Q(s_{t+1}, a_{t+1}; \theta) & \text{if non-terminal states}
      \end{cases}
    $ \;
    Perform gradient descent step on $(y_j - Q(s_t, a_t ; \theta))^2$ \;
    Increment $j$\;
  }
}
\caption{Deep Q-learning}
\label{alg:dqlearning}
\end{algorithm}

Just as \parencite{mnih_playing_nodate} this implementation has the following trock to reduce the number of computations needed. The Q-function maps a state action pair to a scalar value estimates of the Q-value, the most obvious implementation would let the action be part of the input to the function. However, such implementation would require $\actionspace$ number of look ups, at each step, since each action in the action space must be used for evaluation. Instead the Q-function maps the state space to a scalar value for each action, just as described in section \ref{sec:rl_theory}: $Q: \statespace \mapsto \R^{\mid \actionspace \mid}$. 

The solutions used the following hyper parameters and architecture decisions: The Deep Neural network consists of an input layer (size of state space). Then  2 fully connected layers of width $256$ using rectified linear units as activation functions. Finally the network uses a linear output layer for its predictions. The output layer is of the same size as the action space. The learning rate, $\alpha=0.0005$ and I let epsilon be decremented after each update by: $\epsilon_i = \max (\epsilon_{i-1} \cdot 0.9999, 0.01$), where $0.01$ is the minimum exploration that can be done. The replay buffer has a capacity of 1 million rows, and the mini batches used for updating the parameters is $64$ rows. I scale the state space so that each row variable approximately has mean zero and standard deviation of 1 when doing the batch training. This helps performance, and in general accepted as being an important step for getting good performance out of neural networks \parencite{goodfellow_deep_2016}. The impatience parameter is set to $\gamma=0.99$, using a standard value. I make the parameter $\beta_L$ a part of the state space, drawing a random value (uniformly) in the interval $[0.2, 6.0]$ at the beginning of each episode, letting the agent navigate through the environment with given preference for leisure. This is done, so only a single solution of the model is necessary for estimating the parameter later. I scale the rewards (so they are approximate zero mean and have a standard deviation of 1, conditional on the $\beta_L$ value. Again this is done to improve the training performance and to do introspection; it becomes possible to see if there is any trend in the training performance. The algorithm is trained for 3000 episodes. Figure \ref{fig:training_performance_simple_model}  (left plot) shows the training performance of the Deep Q-learning algorithm. The plot shows the cumulative rewards over the life cycle of the agents for each episode. The performance begins at around 2.5 and ends at an average of $7.9$ as the asymptotic value. The performance seems to reach this asymptotic level at around 1000 episodes. I use the Adam optimizer for the gradient descent step when updating the weights.

\begin{figure}[ht]
\begin{subfigure}{.5\textwidth}
  \centering
  \includegraphics[width=1\linewidth]{figures/dqn_training_performance_simple_model.png}
  \caption{DQN Training Performance}
  \label{fig:dqn_training_performance_simple}
\end{subfigure}%
\begin{subfigure}{.5\textwidth}
  \centering
  \includegraphics[width=1\linewidth]{figures/ddqn_training_performance_simple_model.png}
  \caption{Double DQN Training Performance}
  \label{fig:ddqn_training_performance_simple}
\end{subfigure}
    \caption{Comparing training perfance of Deep Q-Learning and Double Deep Q-Learning}
    \label{fig:training_performance_simple_model}
\end{figure}

Figure \ref{fig:dqn_solution_beta2} and \ref{fig:dqn_solution_beta4} compares the performance of the algorihtm to two different benchmarks. One benchmark is the when the $\beta_L = 2$ another $\beta_L=4$. These differences in preferences will yield different optimal policies conditional on the $\beta_L$. I compare the policy chosen by the agent with different deterministic policies. Figure \ref{fig:dqn_solution_beta2} (left plot) shows either using $0, 37, 45$ hours pr. week as comparisons and an agent picking randomly. All agents have been simulated for 300 episodes, yielding the plots being averages, and the standard deviations are calculated on an age basis for each agent. It should also be noted that $\epsilon$ (the exploration ratio of the agent) is set to 0, such that the agent now only greedily explores the environment. Comparing this to the DQN agent I find that it's reasonable to assume that the DQN agent has learned to navigate the environment. A small dip from around age 55 can be observed, but I believe this does not make the feat any less impressive, and furthermore I believe that the solution fairly accurately approximates the correct optimal policy. Looking to the the right hands side plot, the variance of the utility (1 standard deviation). Again the best benchmark for $\beta_L = 2$ an agent that works 45 hours a week, seem have a overlap even in the last periods, where they diverge a little, allowing me to conclude, that this differences in performance is very small. Figure \ref{fig:dqn_solution_beta4} shows the same plot just for $\beta_L = 4$. Again the DQN agent fairly accurately approximates the optimal policy, with even less variance than observed in with $\beta_L = 2$. One thing to note is, that a small hump can be observed for the DQN agent at around age 28, where it has found a way to beat the deterministic policy. Again this might just be variance of the simulations. 


\begin{figure}[ht]
\begin{subfigure}{.5\textwidth}
  \centering
  \includegraphics[width=1\linewidth]{figures/dqn_model1_beta_2_solution_benchmark_paths.png}
  \caption{Simulated Paths}
  \label{fig:dqn_solution_beta2_path}
\end{subfigure}%
\begin{subfigure}{.5\textwidth}
  \centering
  \includegraphics[width=1\linewidth]{figures/dqn_model1_beta_2_solution_benchmark_variance.png}
  \caption{Variance of Paths}
  \label{fig:dqn_solution_beta2_var}
\end{subfigure}
    \caption{Deep Q-Learning solution vs. benchmark $(\beta_L = 2)$}
    \label{fig:dqn_solution_beta2}
\end{figure}

\begin{figure}[ht]
\begin{subfigure}{.5\textwidth}
  \centering
  \includegraphics[width=1\linewidth]{figures/dqn_model1_beta_4_solution_benchmark_paths.png}
  \caption{Simulated Paths}
  \label{fig:dqn_solution_beta4_path}
\end{subfigure}%
\begin{subfigure}{.5\textwidth}
  \centering
  \includegraphics[width=1\linewidth]{figures/dqn_model1_beta_4_solution_benchmark_variance.png}
  \caption{Variance of Paths}
  \label{fig:dqn_solution_beta4_var}
\end{subfigure}
    \caption{Deep Q-Learning solution vs. benchmark $(\beta_L = 4)$}
    \label{fig:dqn_solution_beta4}
\end{figure}


\subsection{Double Deep Q-learning}

Deep Q-learning is good starting point for an learning algorithm using non linear function approximation, however certain properties of the algorithm is problematic. This is mainly due to the fact that Deep Q-learning can tend to be overoptimistic, and that in a problematic way \parencite{van_hasselt_deep_2015}. In general there exists two ways in which overoptimism can be good or at least not detrimental to performance. 1) If the algorithm uniformly overestimates the Q-function for all actions, then this is not associated with any problems, since then taking the max over the action, will lead to the same result had the estimations been correct. In other words performance would not change, however introspection of the algorithm might be harder. 2) It can be a good thing for an algorithm to be optimistic when faced with uncertainty. If an action would lead to observing an unexplored part of the state space, it might be a good thing to be optimistic with regards to exploration, allowing for possible new policies with higher yielding returns over the episode. Deep Q-learning however does not adhere to any of these properties, instead it usually overestimates the value of the action it has taken due to the max operator in the value estimation: $R_{t+1} + \gamma \underset{A_{t+1}}{\argmax} Q(S_{t+1}, A_{t+1})$. An simple extension \parencite{van_hasselt_deep_2015} is to de-couple the prediction with the evaluation. This can be done by having two different sets of weights $\theta$ and $\tilde{\theta}$, where one is used for evaluation and one is used for the policy. This can be presented the following way. The Q-network target:

\begin{equation}\label{eq:dqn_target}
    Y_t^{Q} = R_{t+1} + \gamma Q(S_{t+1}, \underset{a}{\argmax} (S_{t+1}, a ; \theta) ; \theta)
\end{equation}

and the Double DQN target is calculated:

\begin{equation}\label{eq:ddqn_target}
    Y_t^{DoubleQ} = R_{t+1} + \gamma Q(S_{t+1}, \underset{a}{\argmax}Q(S_{t+1}, a; \theta); \tilde{\theta}) 
\end{equation}

This alleviates the problem of overestimating the values taking by the algorithm for the following reason. Consider \eqref{eq:dqn_target}. When calculating the value of the state action pair, one uses the same function to choose the action (through the $\argmax$ operator, which can cause the overestimation of the state action pair of the chosen action. The equation below \eqref{eq:ddqn_target} instead estimates the value of the position by using a different set of parameters, $\tilde{\theta}$. Following the implementation of \parencite{van_hasselt_deep_2015} instead of training 2 separate networks instead, which would be perfectly decouple, to reduce time one of the network simply inherits the old weights. Which imply $\tilde{\theta} = \theta^{previous}$. Such that, the target estimation is made by the old weights, while the value of greedy policy uses the new weights:

\begin{equation}\label{eq:ddqn_target_final}
    Y_t^{DoubleQ} = R_{t+1} + \gamma Q(S_{t+1}, \underset{a}{\argmax}Q(S_{t+1}, a; \theta_t); \theta_t^{previous}) 
\end{equation}

The full algorithm, which to large extend mirrors the algorithm for Deep Q-learning is as follows, is presented in algorithm \ref{alg:ddqlearning}:

\begin{algorithm}[H]
\SetAlgoLined
 Initialize replay memory $\mathcal{D}$ with capacity $N$\;
 Initialize action-value function $Q$ with random weights: $\theta, \tilde{\theta}$\;
 Initialize memory counter $j \leftarrow 1$\;
 Initialize $k$ when to replace target weights $\tilde{\theta}$\;
 \ForEach {episode $\in \{1, 2, \cdots M \}$}{
  Initialize sequence with an initial state $s_1$. This is drawn randomly.\;
    \For{$t \in \{1, 2, \cdots, T$ \}}{
    With probability $\epsilon$ select a random action $a_t$\;
    Otherwise select $A_t = \underset{a}{\argmax}Q^*(S_t, A_t ; \theta)$\;
    Execute action $A_t$ in environment and observe reward $R_t$ and the new state $S_{t+1}$\;
    Store transition $(S_t, A_t, R_{t+1}, S_{t+1})$ in $\mathcal{D}$\;
    Sample random mini-batch of transitions $(S_t, A_t, A_{t+1}, S_{t+1})$ from $\mathcal{D}$\;
    Set $
      Y_j = \begin{cases}
        R_j & \text{if terminal states} \\
        R_j + \gamma Q(S_{t+1}, \underset{a}{\argmax} (Q_t, S_{t+1}; \theta_t); \tilde{\theta_t}) & \text{if non-terminal states}
      \end{cases}
    $ \;
    Perform gradient descent step on $(Y_j - Q(S_t, A_t ; \theta))^2$ w.r.t $\theta$ \;
    If $j$ is divisible by $k$ replace target weights $\tilde{\theta} \leftarrow \theta$\;
    Increment $j$ \;
  }
}
\caption{Double Deep Q-learning}
\label{alg:ddqlearning}
\end{algorithm}

Just as with the Deep Q-learning implementation i let the Q-function map from: $\statespace \mapsto \R^{\mid \actionspace \mid}$, to reduce the number of computations, when taking the max over the Q-function. 


For the Double DQN agent I in general use the same hyper parameters used for the DQN agent: The network architecture is again an input layer of the size of the state space, followed by two fully connected layers with Rectified Linear units as activation functions ending with an output layer with linear activation of same size as the action space. Again i let the learning rate be $\alpha = 0.0005$ and the $\epsilon_i = \max (\epsilon_{i-1} \cdot 0.9999, 0.01)$, such that exploration is performed through out the training period. Again the replay buffer has a capacity of a million rows and i use mini batches of size 64 to perform stochastic gradient descent on the weight of the neural network approximating the Q-function. I transform the states such that they are mean zero and a standard deviation of 1. The same is true for the rewards conditional on the $\beta_L$ value. The $\beta_L$ value is just as with $DQN$ and value function iteration added to the state space, allowing for a single solution of the model. I uniformly draw $\beta_L$ from the interval $[0.2, 6]$ and sets this to be part of the environment in the beginning of each episode. When using gradient descent on the weights I use the Adam optimizer. I let the target network inherit the old weights every 100'th iteration of the algorithm. The algorithm is trained for 3000 episodes. Figure \ref{fig:training_performance_simple_model} (right plot) shows the performance. Very similar to the DQN I find that the performance stabilized at around 1000 episodes, and reaches an asymptotic performance of $8.22$, which is slightly higher compared to the DQN agent which had an asymptotic value of $7.92$.

\begin{figure}[ht]
\begin{subfigure}{.5\textwidth}
  \centering
  \includegraphics[width=1\linewidth]{figures/ddqn_model1_beta_2_solution_benchmark_paths.png}
  \caption{Simulated Paths}
  \label{fig:ddqn_solution_beta2_path}
\end{subfigure}%
\begin{subfigure}{.5\textwidth}
  \centering
  \includegraphics[width=1\linewidth]{figures/ddqn_model1_beta_2_solution_benchmark_variance.png}
  \caption{Variance of Paths}
  \label{fig:ddqn_solution_beta2_var}
\end{subfigure}
    \caption{Double Deep Q-Learning solution vs. benchmark $(\beta_L = 2)$}
    \label{fig:ddqn_solution_beta2}
\end{figure}

\begin{figure}[ht]
\begin{subfigure}{.5\textwidth}
  \centering
  \includegraphics[width=1\linewidth]{figures/ddqn_model1_beta_4_solution_benchmark_paths.png}
  \caption{Simulated Paths}
  \label{fig:ddqn_solution_beta4_path}
\end{subfigure}%
\begin{subfigure}{.5\textwidth}
  \centering
  \includegraphics[width=1\linewidth]{figures/ddqn_model1_beta_4_solution_benchmark_variance.png}
  \caption{Variance of Paths}
  \label{fig:ddqn_solution_beta4_var}
\end{subfigure}
    \caption{Double Deep Q-Learning solution vs. benchmark $(\beta_L = 4)$}
    \label{fig:ddqn_solution_beta4}
\end{figure}

Figure \ref{fig:ddqn_solution_beta2} and \ref{fig:ddqn_solution_beta4} shows the performance of the Deep DQN agent comparing it to two different preferences of leisure: $\beta_L = 2$ and $\beta_L = 4$. Four agents is used for comparison: An agent that chooses actions randomly, and three agents that in an deterministic fashion works $0, 37$ or $45$ hours a week. The results of the Double DQN agent is comparable to those found within the other agents (VFI and DQN). Figure \ref{fig:ddqn_solution_beta2} shows the performance where the preference for leisure $\beta_L = 2$. Again the best deterministic policy is the 45 hour agent, which is very close to what the DQ agent chooses, except for the last 5 years of the agent's life cycle where the utility is slightly lower. Comparing the variance of the utility on right hand side plot of figure \ref{fig:ddqn_solution_beta2}, there is overlap of the simulations of the Double DQN agent and the 45 hours a week agent. Figure \ref{fig:ddqn_solution_beta4} compares the Deep DQN agent to the agents when the preference for leissure $\beta_L = 4$. Here again the optimal policy pretty closely mirrors the best deterministic agent, the one which does not work, with a clear overlap presented in figure on right hand side.


%\section{Estimation of Model}

Only a single parameter of the model needs to be estimated: $\beta_L$. Again, just as when tuning the parameters driving the wage indirect inference is used. The parameters are tuned the following way. Simulate $N$ individuals. Find the relevant moments of supplied working hours. Find the mean squared distance between the actual supplied working hours and the simulated. Continue the process until convergence!

Usually when tuning a dynamic structural model, using simulated methods of moments one would supply the parameters not as part of the state space. The reason for this, is that as the size of the state space expands, it gets exponentially more expensive to solve the model. So the algorithm would go: for a given set of parameters need to be estimated, supply a random initialization. Solve the model for given parameters. Use the solved model to calculate the moments. Use numerical optimization to minimize given optimization problem.

\begin{figure}[ht]
\begin{subfigure}{.5\textwidth}
  \centering
  \includegraphics[width=1\linewidth]{figures/dqi_model1_beta_4_solution_benchmark_paths.png}
  \caption{Simulated Paths}
  \label{fig:dqi_solution_beta4_path}
\end{subfigure}%
\begin{subfigure}{.5\textwidth}
  \centering
  \includegraphics[width=1\linewidth]{figures/dqi_model1_beta_4_solution_benchmark_variance.png}
  \caption{Variance of Paths}
  \label{fig:dqi_solution_beta4_var}
\end{subfigure}
    \caption{Value Function Iteration solution vs. benchmark $(\beta_L = 4)$}
    \label{fig:dqi_solution_beta4}
\end{figure}

\iffalse

\begin{figure}
    \centering
    \includegraphics[scale=0.4]{figures/dqi_model1_estimation_Beta_L}
    \caption{Estimation of $\beta_L$ using Value Function Iteration}
    \label{fig:dqi_model1_optimal_beta_L}
\end{figure}

\begin{figure}
    \centering
    \includegraphics[scale=0.4]{figures/dqn_model1_estimation_Beta_L}
    \caption{Estimation of $\beta_L$ using Deep Q-Learning}
    \label{fig:dqi_model1_optimal_beta_L}
\end{figure}

\begin{figure}
    \centering
    \includegraphics[scale=0.4]{figures/ddqn_model1_estimation_Beta_L}
    \caption{Estimation of $\beta_L$ using Double Deep Q-Learning}
    \label{fig:dqi_model1_optimal_beta_L}
\end{figure}

\fi

%\section{Model 1 Results}

I arbitrarily choose value function iteration when investigating the results of the model, since all the solution methods yielded identical results. 5000 agents is simulated using the optimal $\beta_L = 3.43$. Relevant summary statistics can be calculated over the 5000 simulations. The statistics can be used to infer whether or not the model fit the data and compare to what is expected from the real world. Looking to figure \ref{fig:dqi_model1_average_path_sim_vs_empirical} the empirical number of hours supplied by women \textbf{LIGEF15} is compared to the simulated. Again only women actually in the labour force  ($H>0$) is considered. The fit of the data does seem reasonable. In general the simulated data set slightly overestimates the number of supplied hours. Especially around age 30 does the simulated number of supplied hours seem to be higher than the empirical. The simulated data show heterogeneity over the life cycle, where especially the later stages of the life cycle show higher variance. This could very well stem from the idiosyncratic wage path increasing over time. Considering the estimation method relied on the number of supplied hours, it is not surprising that this is where the model performs the best. 


\begin{figure}
    \centering
    \includegraphics[scale=0.4]{figures/dqi_model1_estimation_labour_supply.png}
    \caption{Average Number of Supplied Working Hours - Empirical Vs. Simulated}
    \label{fig:dqi_model1_average_path_sim_vs_empirical}
\end{figure}

Figure \ref{fig:dqi_model1_fraction_in_workforce} shows the participation rate of women in the workforce in the simulated data set. From here it is very clear, that the model underestimates the number of women working! From age 25, 80 \% of the women is out of the workforce, followed by a slow decline in participation going forward to retirement. This does obviously not correspond to what can be observed in the real world. Even though it is not possible to get access to the exact number of women out of the labour force conditional on age, it can be approximated. Using \textbf{FOLK1B} supplied by Statistics Denmark the number of women in the age (15 to 60) can be found. Using that data the number of relevant women (considering the age) is around 1.664 million people. The number of women not working for relevant reasons can be summarized to: women not working due to: \textit{working in the home}, \textit{Studying}, \textit{Other people outside the labour force}. These add up to: $15 + 161 + 66 = 242$ thousand people or $0.242$ million people. The number of women in the relevant age group outside the labour force can be approximated to the number given in equation \ref{eq:women_outside_the_labour_force}

\begin{equation}
    \label{eq:women_outside_the_labour_force}
    \textbf{Women outside the labour force} = \frac{0.242}{1.664} = 0.145 = 14.5 \% \approx 15 \%
\end{equation}

The estimation of $\beta_L$ did not take this measure into account, which might be a way to improve the results.

\begin{figure}
    \centering
    \includegraphics[scale=0.4]{figures/dqi_model1_women_in_labour_force_fraction.png}
    \caption{Fraction of Women in The Labour Force}
    \label{fig:dqi_model1_fraction_in_workforce}
\end{figure}

Both figure \ref{fig:dqi_model1_birth_onset} and figure \ref{fig:dqi_model1_child_vs_no_child_30} is inspired by the article of \textcite{kleven_children_2019}. In the article the authors show how participation of a woman is affected when she gives birth to a child. In the paper they find that women,  when giving birth to a child on average use 20 \% time less on work, compared to the husband, immediately after the birth of the child. The number of supplied hours slowly return to steady state with a long run penalty of around 6 \%. This is not the case for the simulated household. Getting a child, do seem to decrease the number of hours supplied however, the difference is indistinguishable from the mothers not getting children. This is clear in figure \ref{fig:dqi_model1_child_vs_no_child_30}, where first time mothers of age 30 is compared with women that have no children. The paths are indistinguishable, where the paper by \textcite{kleven_children_2019} would suggest a decrease of about 20 percent immediately after the birth (at age 31). These figures do not sort out women outside the labour force. I conclude the model is insufficient to capture the real world, and I move on to extend this original model.


\begin{figure}[ht]
\begin{subfigure}{.5\textwidth}
    \centering
    \includegraphics[width=1\linewidth]{figures/dqi_single_child_vs_no_child_model1.png}
    \caption{First time mothers vs. Women with no children}
    \label{fig:dqi_model1_child_vs_no_child_30}
\end{subfigure}%
\begin{subfigure}{.5\textwidth}
    \centering
    \includegraphics[width=1\linewidth]{figures/women_supplied_hours_dqi_model1_birth_onset.png}
    \caption{hours compared to birth onset.}
    \label{fig:dqi_model1_birth_onset}
\end{subfigure}
    \caption{Impact of Children}
    \label{fig:model_impact_children}
\end{figure}



\section{An extended Model}

After asserting that a deep reinforcement learning agent can navigate the state space easily, model is extended to address the issues found in the simple model presented in section \ref{sec:model1}. The first issue that should be addressed is that women should not permanently withdraw from the labour force. Second the log-function used for the sub-utilities caps the utility very hard, which implies that for very small perturbations of $\beta_L$ makes the agent substitute into one of the extremes. The agent end up working either 45 hours exclusively or working 0 hours exclusively. They agent should ideally be more balanced, implying the functional form of the sub-utilities should have a steeper curvature.

\begin{equation}
    U_t = \beta_L L_{t}^\zeta + \beta_Y Y_t^{\zeta}
\end{equation}

The fertility process should also be extended. I now assume that each household can have a maximum of four kids. The environment tracks how many kids the family has, and the associated age. Still in each period the number of children in the household can increment by one.

\begin{equation}
    K_{t+1} = K_t+ \psi_t, \qquad \psi_t \mid Q_t \sim Bernoulli (p_\psi(Q_t))
\end{equation}

However i now let $B^\top$ be a vector of size 4 equal to the maximum number of children. Furthermore i let the vector $C^\top$ denote the individual age of the children also of size 4. This implies that for index $i$ of vector $B$ corresponds to the $k$'th child, index $i$ of vector $C$ implies the same child's corresponding age.

\begin{equation}
    B_t[i]  = \begin{cases}
        1 & \text{if }  i \leq K_t \\
        0 & \text{else}
    \end{cases}
\end{equation}

and the age vector is defined as:

\begin{equation}
    C_{t+1} = C_t + B_{t+1}
\end{equation}

So that the age of the children only increments if and only if the child is born.

Finally the utility part of the model needs to be adjusted to take into account the number of hours used pr. child. Let $J$ denote the number of hours used on children. Using the numbers found by \parencite{ekert-jaffe_time_2015} children below age 3 takes 10 hours pr week, children in the age 3 up to age 15 takes 3.5 hours pr. week, and children above takes 0 hours of leisure pr. week for women. 

\begin{equation}
    J_t = \sum_{i=1}^4 \begin{cases}
        0.0 & \text{if } B_t[i] = 0 \\
        10.0 & \text{if } B_t[i] = 1 \land C_t[i] < 3 \\
        3.5 & \text{if } B_t[i] = 1 \land 3 \geq C_t[i] < 16 \\
        0 & \text{if } B_t[i] = 1 \land  16 \geq  C_t[i] 
    \end{cases}
\end{equation}

Another thing to address is, that empirically women tend to work less when they are young (in the age of 18 to 30), gradually increasing the number of working hours as shown in figure \ref{fig:dqi_model1_average_path_sim_vs_empirical}. This is probably a consequence of education. Looking to figure \ref{fig:educ_empirical}, it could appear to be a consequence of women doing full time education, not having the time to work (more than a few hours a week) that could explain the result. To address this problem I add to the model an education parameter. In this setup the education does not add to higher salary instead it only goes in the model as additional time spent. Education is assumed to be exogenous, and over the life cycle the women will with some probability be done with education, not to return ever again to the education system. Using \textbf{FOLK1A} from Statistics Denmark i find the total number of women in a given age group. The total number of women for a given age group doing full time education is found in the source \textbf{UDDAKT10} also from Statistics Denmark. I fit this with a linear regression and find a path linearly decreasing the number of women doing full time education This can be seen in Figure \ref{fig:prob_educ_full_time}. I model the education the following way:

\begin{figure}[ht]
\begin{subfigure}{.5\textwidth}
  \centering
  \includegraphics[width=1\linewidth]{figures/total_women_vs_education.png}
  \caption{FTE vs. Total Number of Women}
  \label{fig:educ_empirical}
\end{subfigure}%
\begin{subfigure}{.5\textwidth}
  \centering
  \includegraphics[width=1\linewidth]{figures/prop_women_doing_full_time_education.png}
  \caption{Fraction of Women in FTE vs. Linear Fit}
  \label{fig:prob_educ_full_time}
\end{subfigure}
    \caption{Women and Full Time Education (FTE)}
    \label{fig:educ_women}
\end{figure}

\begin{equation}
    E_{t+1} = E_{t} \cdot \iota_t, \qquad \iota_t \mid Q_t \sim  Binomial(p_{\iota}(Q_t))
\end{equation}

So in each period with some probability $E^{prob}_t \equiv p_{\iota}(Q_T)$ the women will end full time education. Both $E$ and $E^{prob}_t$ is part of the state space. Since women do full time education, I assume time used is 37 hours pr. week.

It is assumed for simplicity that human capital accumulation does not depreciate or accumulate when the agent is under full time education:

\begin{equation}
    G_{t+1} = 
    \begin{cases}
        G_t(1 - \delta) + \frac{H_t}{37}, & \text{if } E_t > 0 \\
        G_t & \text{else}
    \end{cases}
\end{equation}


Finally note that the utility is calculated pr. week basis instead of on an annual basis. This is again done, such that the sub utilities, do not squeeze the values too hard, as was the case in model 1, making estimating of the model very hard, allowing small perturbations of $\beta_L$ having to much effect. Compared to the original model $f^M(Q_t)$ now represents the number total income pr. week of the husband rather than the annual total income.

\begin{equation}
    L_t = 24 \cdot 7 - H_t - J_t - E_t \cdot 37
\end{equation}

In Denmark education is coupled with a transfer from the state. This transfer is added as well, such that the household will a transfer of $tr = \frac{6000 DKK \times 12}{52} \approx 1400 DKK$ each week from the government, while the women is studying. This ratio is approximately doubled if the women gets a child during the education \parencite{noauthor_satser_nodate-1}. The total income of the household is equal to:

\begin{equation}
    Y_t = H_t \cdot W_t + f^M(Q_t) + E_t (tr + \mathbf{1} \{ K_t > 0 \} \cdot tr)  
\end{equation}


Summarizing the model; using a lot of the framework from the simple model a couple of extensions is made. The state space is now of 14 dimensions containing $(Q)$ age, $(G)$ human capital, $(Z)$ idiosyncratic wage path, $(K)$ the number of children in at time $t$ in the household, a vector of size four $(B)$ corresponding to the number of children the household contain, a vector of size four $(C)$ containing the age of the individual children, a dummy $(E)$ that indicates if the women is under education, and lastly $E^{prob}$ a number between 0 and 1 that indicates the probability of being enrolled in full time education next period. The action space is slightly tweaked to contain $\{ 0, 15, 25, 37\}$ representing the discreet choices of hours the woman can choose to work. This formulation is closer to that of \textcite{francesconi_joint_2002}, where he only considers the choices \{nonwork, part-time, full-time\}. In the formulation of this extended model, there is distinguished between two types of part time work; 15 and 25 hours a week. The model specification is:

\begin{equation}
    \textbf{State space: }\statespace = \R^2 \times \{0, 1, 2, 3, 4\} \times \{18, 19, \cdots , 60\} \times \{0, 1\}^4 \times \{0, 1, \cdots, 18\}^4 \times \{0,1\} \times [0, 1]
\end{equation}

\begin{equation}
    \textbf{Action space: }\actionspace = \{0, 15, 25, 37\}
\end{equation}

Additionally the model contains the following parameters: $\alpha=, \eta_G = 0.164, \eta_{G^{2}}=0.015, \delta, \sigma_\epsilon = 15.11, \beta_L, \beta_Y = 1, W_{min} = 120, tr = 1600, \zeta=0.5$, using the values of the Mincer equation from the parameter calibration performed earlier! A recursive formulation of the model is presented below:

\begin{align}
    U_t(L_t, Y_t) &= \beta_L L_t^{\zeta} + \beta_Y Y_t^{\zeta}\\
    L_t(K_t, H_t, J_t, E_t) &= 24 \cdot 7 - H_t - J_t - E_t \cdot 37\\
    \log \tilde{W}_t (G_t) &= \alpha + \eta_G G_t + \eta_{G^2} G_t^2 \\
    W_t(\tilde{W}_t, Z_t) &= \max(W_{min} , \tilde{W}_t  + Z_t)  \\
    Y_t(Q_t,H_t, W_t, E_t, K_t) &= 46 \cdot H_t \cdot W_t + f^M(Q_t) +  E_t (tr + \mathbf{1} \{ K_t > 0 \} \cdot tr)\\
    J_t (B_t, C_t) &= \sum_{i=1}^4 \begin{cases}
        0.0 & \text{if } B_t[i] = 0 \\
        10.0 & \text{if } B_t[i] = 1 \land C_t[i] < 3 \\
        3.5 & \text{if } B_t[i] = 1 \land 3 \geq C_t[i] < 16 \\
        0.0 & \text{if } B_t[i] = 1 \land  16 \geq  C_t[i] 
    \end{cases} \\
    B_t[i] (K_t) &= \begin{cases}
        1 & \text{if }  i \leq K_t \\
        0 & \text{else}
    \end{cases}
\end{align}

Law of motion:

\begin{align}
    Q_{t+1}(Q_t) &= Q_t \\
    K_{t+1}(K_t, Q_t)  &= K_{t} + \psi_t, \qquad \psi_t \mid Q_t \sim Bernoulli(p_\psi(Q_t))  \\
    Z_{t+1}(Z_t) &= Z_t + \epsilon_t, \qquad \epsilon_t \sim \ndist(0, \sigma_\epsilon)\\
    G_{t+1}(G_t, H_t, E_t) &= 
    \begin{cases}
        G_t(1 - \delta) + \frac{H_t}{37}, & \text{if } E_t > 0 \\
        G_t & \text{else}
    \end{cases} \\
    C_{t+1}(C_t, B_{t+1}) &= C_{t} + B_{t+1} \\
    E_{t+1}(E_t, Q_t) &= E_t \cdot \iota_t, \qquad \iota_t \mid Q_t \sim Binomial(p_\iota(Q_t))
\end{align}




\clearpage

%\printbibliography

\clearpage

%\section*{Appendix}

\listoffigures
\listoftables

\newpage

\begin{figure}
    \centering
    \includegraphics[scale=0.3]{figures/kleven_10_years_impact.png}
    \caption{Event Graphs of \textcite{kleven_children_2019}}
    \label{fig:event_graphs_kleven}
\end{figure}

\begin{figure}
    \centering
    \includegraphics[scale=0.4]{figures/fertility_women_empirical.png}
    \caption{Fertility of Women}
    \label{fig:fertility_by_women}
\end{figure}



\end{document}
