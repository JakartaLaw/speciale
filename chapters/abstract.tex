\begin{abstract}
This is the abstract.


I use data of 11 stocks for the past 30 years. First the data is presented, using graphs and summary statistics, for the individual stocks. Next I present a structural causal model that attempts to model stock returns under a regime of structural breaks. I identify the parameters necessary for sampling from the structural model. Using the real data collected, I use log-likelihood estimation to find the parameter values. I simulate a data set of $11$ stocks with $2.000.000$ rows containing slightly less than $10.000$ structural breaks. I move onto present three different algorithms for predicting the Sharpe ratio in a given time period. These being a deep learning model called LSTM, and two simpler algorithms using rolling window estimations of Sharpe ratios. I discuss the different properties, and how each model should be trained and tuned. Using the simulated data set i compare the three different algorithms and their capacity to accurately predict the Sharpe ratio of individual stocks. Using the best model i investigate the performance on the real data set comparing to different benchmarks. Using not only summary statistics of the different strategies for stock picking, I also plot the counter factual investment strategies, and make a Monte Carlo simulation for comparing the performance of the different strategies.
\end{abstract}
