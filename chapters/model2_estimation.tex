\section{Estimation of extended Model}

Just as it was the case with the simpler model, only a single parameter $\beta_L$ needs to be estimated. Again, just as was the case before with the simpler model, I have extended the state space with $\beta_L$. I use a grid search in the range $\beta_L \in [0, 60]$, simulating $N=800$ observations, calculating the objective function. I use the same seed in each iteration of the optimization process. To address the short comings of the simple model i expand the objective function of the optimization problem. In the initial model to many women choose not to be part of the labour force yielding unrealistic results. Therefore the objective function now extends to two broad goals: Have the right number of women be out of the labour force (around15 \%) and let fit the curve of number of working hours for women. The first objective, \textit{objective 1}, is formulated as: 

\begin{equation}
    \text{objective 1} = \lsp\frac{\sum_{i=1}^{N} \sum_{t=1}^T \mathbf{1}\{H_{i,t} = 0\}}{ N \cdot (Q_{\max} - Q_{\min} )} - 15\% \rsp^2
\end{equation}

The second objective is formulated the same way as it was when first estimating the simpel model, which correspond the conditional expectation of supplied number of working hours conditional on the age and of being in the labour force $ \E[H \mid Q=q, H>0]$ Now the desired outcome is for this to be true for all ages from 18 to 60:


\begin{equation}
    \text{Objective 2} = \sum_{t=1}^{T} \lsp \frac{1}{\sum_{i=1}^{N} \mathbf{1}\{H_{i,t} > 0\}}\sum_{i=1}^N (H_{i,t})\rsp^2
\end{equation}

Now this description of the objectives can be translated into a more formal formulation. Considering this that there $\text{\# moments}  = 1 + (Q_{\max} - Q_{\min}) = 1 + 60 - 18 = 43$. Now the weighting of these moments would usually be done by some weight matrix $W$ in a formula looking like: $[\hat{m} - \tilde{m}]^{\top} W^{-1} [\hat{m} - \tilde{m}]$, where $\tilde{m}$ is a vector of empirical moments and $\hat{m}$ is a vector of simulated moments allowing for the methods of moments estimation. The choosing of the weight matrix is application specific. (XXX HECKMAN 2015) suggests the choice in that setting will be the variance of empirical moments $\tilde{m}_j$, on the diagonal of the weight matrix, letting the $j'th$ moment correspond to the $j'th$ weight. This is however not possible in this application, due to the fact, that I only have access to aggregated data from statistics Denmark. Instead i manually chooses scales such that \textit{objective 1} is equally weighted to \textit{objective 2}. This first and foremost requires a scaling of the moments. And second of all this requires a re weighting since there is 42 moments composing \textit{objective 2} and only a single moments composing \textit{objective 1}. Note here that since the distance between the empirical and the simulated moment is squared, the scaling must be performed before the squaring. Finally it should be noted that since the weight matrix is chosen as it is, a transformation can be performed such that instead of matrix product it can be considered a sum: $\sum_{j = 1}
^{43} ( \hat{m}_j - \tilde{m}_j )^2$.

The objective function does in other words look as, where it's implied that $N$ agents is simulated conditional on a value of $\beta_L$:

\begin{equation}
   \text{Objective}(\beta_L) =  \lsp 37 \cdot \lp\frac{\sum_{i=1}^{N} \sum_{t=1}^T \mathbf{1}\{H_{i,t} = 0\}}{ N \cdot (Q_{\max} - Q_{\min} )} - 15\% \rp \rsp^2 + \frac{1}{Q_{\max} - Q_{\min}}\sum_{t=1}^{T} \lsp \frac{1}{\sum_{i=1}^{N} \mathbf{1}\{H_{i,t} > 0\}}\sum_{i=1}^N (H_{i,t})\rsp^2
\end{equation}

XXX INDSÆT BILLEDE

The estimated value of $\beta_L = 24.49$. Figure (XXX ESTIMATION) shows the grid serach, and finds that optimization problem, does seem to have a fairly unique minima at $\beta_L = 24.49$, suggesting that the range of the grid is adequate for the estimation problem. I take the $\log (\cdot)$ of the mean squared error to better represent it in figure XXX, allowing for not showing the less extreme tails.