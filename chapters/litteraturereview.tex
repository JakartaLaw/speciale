\section{Review of Literature}\label{sec:lit_review}

There exists a deep literature on both labour supply and fertility.
As this paper tries to investigate the relationship between women's labour supply and fertility by formulating a dynamic  structural model, the focus will lie on literature that has the same scope. The paper that has been the main inspiration for this paper has been work by \textcite{francesconi_joint_2002}, where he proposes a joint dynamic model of fertility and labour supply of married women. Francesconi formulates a model with a joint decision of fertility and labour supply. Labour supply is considered in his formulation split into three choices: not work, work part-time and work full-time. Furthermore Francesconi proposes a model of human capital accumulation as a function of the labour supply. Additionally he assumes a budget constraint of all income in period $t$ should be consumed in period $t$, and the husbands income and labour supply is modelled to follow an exogenous process.  main findings of the paper is that a clear relationship between earnings ability and preference for work, where women with highest earnings profiles negatively, has the lowest marginal utility of children \parencite{francesconi_joint_2002}.

Work has also been done in extending the basic framework proposed by Francesconi that models joint decision between fertility and labour supply.  \textcite{adda_career_2011} extends the life cycle model by allowing for savings, and more sophisticated skill atrophy processes,  \textcite{gayle_life-cyle_2006} allows for the participation rate of women to be continuous and \textcite{keane_role_2010} focuses on the marriage market while still allowing for fertility and labour supply being part of the choice set. \textcite{adda_career_2011} suggest that fertility might be falling in developed countries due to significant cost to the careers and future earnings of women. They find that the cost of career interruptions as a consequence of children, is non-linear over the career cycle and has the biggest impact around mid-career. They also find that children has influence on career planning, and the effect of planning to have children, affects the career even before the first child. \textcite{gayle_life-cyle_2006} employ a semi-parametric approach to a panel data setting. Inline with \textcite{francesconi_joint_2002}, \textcite{gayle_life-cyle_2006} finds that having children is less desirable for women on high income trajectories. \textcite{keane_role_2010} finds that differences in skill rental price between black and white women can explain the number of teenage pregnancies, implying again that a relationship between fertility and career trajectories is present.

The joint decision between fertility and labour supply has been investigated before \textcite{francesconi_joint_2002} investigated the problem by proposing a dynamic structural model. Both \textcite{moffitt_estimation_1984} and \textcite{hotz_empirical_1988} utilizes an closed form approach to the problem of fertility and labour supply to estimate an econometric model. \textcite{moffitt_estimation_1984} employs a cross sectional approach and finds that he is able to better fit the bimodal distribution of working hours. While controlling for number of children he finds that more children in general imply women work less. \textcite{hotz_empirical_1988} makes the interessting addition to their model, that fertility is only somewhat a choice of the household, i.e. with some probability the fertility outcome will diverge from the household's expectation. They find that maternal time required for a newborn child is 660 hours per year, decreasing geometrically as the child ages. They also find that children do not reduce a woman's labour supply 1-to-1, rather the time spend on children will be taken from other activities as well.

Looking to the effect of children on labour supply has been investigated by \parencite{angrist_children_1996} finding that, the effect of children on labour supply disappear as the child turns 13. The paper emphasizes the causal link ( using IV-estimation) between fertility and labour supply and finds that fertility only have a small effect on the labour supply of women. Considering the effect of children  \parencite{altug_effect_1998} employs a semi-parametric approach to estimating the effect of experience on female labour supple, controlling for children. They find that the impact of children on women's participation rate is ambiguous, but for married women, additional children increases number of hours worked \parencite{altug_effect_1998}.