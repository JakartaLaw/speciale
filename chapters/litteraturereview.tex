\section{Review of Literature}\label{sec:lit_review}

There exists a deep literature on both labour supply and fertility.
As this paper tries to investigate the relationship between women's labour supply and fertility by formulating a dynamic  structural model, the focus will lie on literature that has the same scope. The paper that has been the main inspiration for this paper has been work by \textcite{francesconi_joint_2002}, where he proposes a joint dynamic model of fertility and labour supply of married women. Francesconi formulates a model with a joint decision of fertility and labour supply. Labour supply is considered in his formulation split into three choices: not work, work part-time and work full-time. Furthermore Francesconi proposes a model of human capital accumulation as a function of the labour supply. Additionally he assumes a budget constraint of all income in period $t$ should be consumed in period $t$, and the husbands income and labour supply is modelled to follow an exogenous process.  main findings of the paper is that a clear relationship between earnings ability and preference for work, where women with highest earnings profiles negatively, has the lowest marginal utility of children.

Work has also been done in extending the basic framework proposed by \textcite{francesconi_joint_2002} that models joint decision between fertility and labour supply.  \textcite{adda_career_2011} extends the life cycle model by allowing for savings, and more sophisticated skill atrophy processes,  \textcite{gayle_life-cyle_2006} allows for the participation rate of women to be continuous and \textcite{keane_role_2010} focuses on the marriage market while still allowing for fertility and labour supply being part of the choice set. \textcite{adda_career_2011} suggest that fertility might be falling in developed countries due to significant cost to the careers and future earnings of women. They find that the cost of career interruptions as a consequence of children, is non-linear over the career cycle and has the biggest impact around mid-career. They also find that children has influence on career planning, and the effect of planning to have children, affects the career even before the first child. \textcite{gayle_life-cyle_2006} employ a semi-parametric approach to a panel data setting. Inline with \textcite{francesconi_joint_2002}, \textcite{gayle_life-cyle_2006} finds that having children is less desirable for women on high income trajectories. \textcite{keane_role_2010} finds that differences in skill rental price between black and white women can explain the number of teenage pregnancies, implying again that a relationship between fertility and career trajectories is present.

The joint decision between fertility and labour supply has been investigated before \textcite{francesconi_joint_2002} investigated the problem by proposing a dynamic structural model. Both \textcite{moffitt_estimation_1984} and \textcite{hotz_empirical_1988} utilizes an closed form approach to the problem of fertility and labour supply to estimate an econometric model. \textcite{moffitt_estimation_1984} employs a cross sectional approach and finds that he is able to better fit the bimodal distribution of working hours. While controlling for number of children he finds that more children in general imply women work less. \textcite{hotz_empirical_1988} makes the interessting addition to their model, that fertility is only somewhat a choice of the household, i.e. with some probability the fertility outcome will diverge from the household's expectation. They find that maternal time required for a newborn child is 660 hours per year, decreasing geometrically as the child ages. They also find that children do not reduce a woman's labour supply 1-to-1, rather the time spend on children will be taken from other activities as well.

Looking to the effect of children on labour supply has been investigated by \textcite{angrist_children_1996} finding that, the effect of children on labour supply disappear as the child turns 13. The paper emphasizes the causal link (using IV-estimation) between fertility and labour supply and finds that fertility only have a small effect on the labour supply of women. Considering the effect of children  \textcite{altug_effect_1998} employs a semi-parametric approach to estimating the effect of experience on female labour supple, controlling for children. They find that the impact of children on women's participation rate is ambiguous, but for married women, additional children increases number of hours worked.

\textcite{attanasio_explaining_2008} proposes a structural model, that investigates the extensive margin of women that work, leading them to a formulation where women can either work or not work. Their main addition to their model is letting the households borrow, separating them from \textcite{francesconi_joint_2002} among other earlier studies, that did not allow for such behavior. Human capital is accumulated when women enters the work force. They investigate 3 cohorts (Dole, Clinton and Oprah) as to see the what can account for their different participation rates. Their main finding is that some of the difference can be explained by reduced child-care costs and a reduction in the wage gender gap. A later paper from the same authors \textcite{attanasio_aggregating_2018} they construct a model where they again investigate the participation rate of women. In this formulation they control for family composition and importantly include to "taste-shifters" in the utility function. These are latent variables that us used to explain how the participation rate can change under different circumstances. Their main finding is that heterogeneity of demographics, wealth distribution and the point of the business cycle can explain a lot of the aggregate responses of female labour supply.

\textcite{del_boca_motherhood_2009} investigates uses cross-country European data to investigate the join decision about fertility and labour market participation. They control for personal characteristics as well as childcare system, parental leave system, family allowances as well as part time opportunities. Their main findings include that labour market and social environment do not affect fertility in a significant way. They find however that the institutions that can support women's labour market participation does indeed have impact on the women's labour market participation, especially this effect is the most present in less educated women. These results are somewhat in line with \textcite{haan_can_2009} that investigates the impact of financial incentives on female labour supply by exploiting the variance stemming from the tax and transfer system. They also find that child care subsidies do increase labour supply, however in contrast with \textcite{del_boca_motherhood_2009} they do find that child care subsidies do increase fertility as well.

\textcite{jones_fertility_2008} makes a comparative analysis of the prevalent theories explaining the relationship between fertility and income: whether it is taste or ability that can explain the differences in income. They find that both theories can explain, however the ability hypothesis can only hold under the assumption of a elasticity of substitution between consumption and the number of children. Therefore they conclude that the taste theory is more robust.

\textcite{blundell_female_2016} uses quasi-experiment of the UK tax and welfare reforms of the 1990s and 2000s to investigate women's labour supply. They construct a dynamic model where women can save and accumulate human capital, along with educaiton. They make the women choose education level and their participation rate in the labour market. In contrast with \textcite{francesconi_joint_2002} they do not let fertility be part of the choice space, rather they model it as a random event. They control for demographics etc. They find that labour supply elasticities are generally high (but below 1), except for single mother that seem to have above 1. Another interesting result from the paper is that tax credits do seem to let low-education women into the workforce, however it does not seem to have long term influence on employment or wages for this group. \textcite{eckstein_dynamic_2011} also investigates the discrepancy between lone mothers and married couples by constructing a dynamic life cycle model, their motivation being that while the labour supply of women the last 50 years have had a sharp rising trend, the same cannot be said for unmarried women\footnote{Married women has a gone from 30\% to 60\% employment, where single and divorced women have been at around 70\% employment throughout the sample.}. The authors conclude that the rise in female employment in large part can be explained by the increase in years of schooling and the rise of female wages. They also find that changes in fertility and marital status do not have a big impact. 

Briefly discussing results in a danish context, using Danish data \textcite{kleven_children_2019}. They do find that the long term effects of the arrival of a child reduces earnings by about 20 \%, and the hours worked by 10\% for women, while no effects are notable for men (10 year horizon). The same goes for participation rates which fall about 13 \% and wage rates which falls about 9 \% in the long run for women (10 year horizon). \textcite{jorgensen_life-cycle_2017} investigates the relationship between consumption of non-durable goods and the average number of children, since these follows the same trajectory over the life cycle. He finds that income of the households fall significantly, but in contrast with \textcite{kleven_children_2019} the economic effect is negligible at around 1\% reduction in households with one child compared to households with no children. 

Considering that children can have an effect on the labour force participation of women one might ask why? This can be framed into different ways: women want to spend more time with their kids because it's a more joyful activity. The second option is that some of the time children take up can be considered work, implying less leisure time for women, and to compensate they work less. 