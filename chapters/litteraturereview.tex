\section{Review of Literature}\label{sec:lit_review}

A rich literature on both labour supply and fertility exists. As this paper tries to investigate the relationship between women's labour supply and fertility by formulating a dynamic  structural model, the primary focus will be on literature that has the same scope. The main inspiration for this dissertation has been the paper by \textcite{francesconi_joint_2002}, where he proposes a joint dynamic model of fertility and labour supply of married women. Labour supply is split into three choices: \textit{not work}, \textit{work part-time} and \textit{work full-time}. Furthermore, Francesconi proposes human capital accumulation to be a function supplied labour. Additionally, he assumes a budget constraint of all income in period $t$ should be consumed in period $t$, and the husband's income and labour supply is modelled to follow an exogenous process. The main findings of the paper is that a relationship exists between earnings ability and preference for work and women with the highest earnings profiles has the lowest marginal utility of children.

Work has also been done to extend the basic framework proposed by \textcite{francesconi_joint_2002}, that models the joint decision between fertility and labour supply.  \textcite{adda_career_2011} extends the lifecycle model by allowing for savings, and more sophisticated skill atrophy processes, \textcite{gayle_life-cyle_2006} allows for the participation rate of women to be continuous and \textcite{keane_role_2010} focuses on the marriage market while still allowing for fertility and labour supply being part of the choice set. \textcite{adda_career_2011} suggests that fertility might be falling in developed countries due to significant costs to the careers and future earnings of women associated with child birth. They find that the cost of career interruptions as a consequence of children is non-linear over the career cycle, and it has the biggest impact around mid-career. They also find that children influence career planning even before the first child is born. \textcite{gayle_life-cyle_2006} employ a semi-parametric approach to a panel data setting. Inline with \textcite{francesconi_joint_2002}, \textcite{gayle_life-cyle_2006} finds that having children is less desirable for women on high income trajectories. \textcite{keane_role_2010} finds that differences in skill rental price between black and white women can explain the number of teenage pregnancies, again implying that a relationship between fertility and career trajectories is present.

The joint decision between fertility and labour supply has been investigated before \textcite{francesconi_joint_2002} proposed his dynamic structural model. Both \textcite{moffitt_estimation_1984} and \textcite{hotz_empirical_1988} utilizes a closed form approach to the problem of fertility and labour supply to estimate an econometric model. \textcite{moffitt_estimation_1984} employs a cross-sectional approach. While controlling for the number of children he finds that more children in general imply women work less. \textcite{hotz_empirical_1988} makes the interesting addition to their model, that fertility is only somewhat a choice of the household, i.e., with some probability the fertility outcome will diverge from the household's expectation. They find that the maternal time required for a newborn child is 660 hours per year, which equates to $12.7$ hours per week, decreasing geometrically as the child ages. They also find that children do not reduce women's labour supply 1-to-1, rather the time spent on children will be taken from other activities as well.

Not all research has lead to the same unambiguous conclusion, namely, that children lead to a reduction in the labour supply of women.   \textcite{angrist_children_1996}  investigates the causal link (using IV-estimation) between fertility and labour supply, and finds that fertility only have a small effect on the labour supply of women. Additionally, they find the effect seems to disappear as the child turns 13.  \textcite{altug_effect_1998} employs a semi-parametric approach to estimate the effect of experience on female labour supply while controlling for children. They find that the impact of children on women's participation rate is ambiguous, but for married women, additional children increases number of hours worked.

Much of the literature on female labour assumes an income constraint, such that households are not allowed to consume what is not yet earned.  \textcite{attanasio_explaining_2008} proposes a structural model, that investigates the extensive margin of women that work, with the addition to their model  of letting households borrow, separating them from \textcite{francesconi_joint_2002} among other earlier studies. Human capital is accumulated when women enter the work force. They investigate 3 cohorts to see the what can account for the different participation rates observed. Their main finding is that some of the differences observed in the cohorts, can be explained by reduced child-care costs and a reduction in the wage gender gap. Attanasio has investigated female labour supply further.  \textcite{attanasio_aggregating_2018} formulates a model where they investigate the participation rate of women. In this formulation they control for family composition and importantly include ``taste-shifters'' in the utility function. These are latent variables that is used to explain how the participation rate can change under different circumstances. Their main finding is that heterogeneity of demographics, wealth distribution and the point of the business cycle can explain a lot of the aggregate responses observed in female labour supply.

The impact of institutions and benefits that support mothers is also investigated in the literature. \textcite{del_boca_motherhood_2009}  uses cross-country European data to investigate the joint decision between fertility and labour market participation. They control for personal characteristics as well as country-specific childcare systems, parental leave systems, family allowances as well as part-time opportunities. Their main findings include that the labour market and social environment do not affect fertility significantly. They do however find that institutions that can support women's labour market participation does indeed have an impact on women's labour market participation. This effect is the most pronounced in less educated women. These results are somewhat in line with \textcite{haan_can_2009} that investigates the impact of financial incentives on female labour supply by exploiting the variance stemming from the tax and transfer system. They also find that child care subsidies increase labour supply. In contrast with \textcite{del_boca_motherhood_2009} they do find child care subsidies to increase fertility as well.

This paper solely focuses on women in relationships. It is not unreasonable to think, that differences between single and married women exists regarding fertility and labour supply. \textcite{blundell_female_2016} uses a quasi-experiment focusing on the UK tax and welfare reforms of the 1990s and 2000s to investigate the labour supply of women. They construct a dynamic model where women can save and accumulate human capital along with education. They make the women choose education level and their participation rate in the labour market. In contrast with \textcite{francesconi_joint_2002} they do not let fertility be part of the choice space, rather they model it as a random event. They control for demographics etc. They find that labour supply elasticities are on average high (but below 1), except for single mother that seem to have above 1. Another interesting result from the paper is, that tax credits do seem to let low-education women into the workforce, however, it does not seem to have long-term influence on employment or wages for this group. \textcite{eckstein_dynamic_2011} also investigates the discrepancy between lone mothers and married couples by constructing a dynamic lifecycle model, their motivation being that while the labour supply of women the last 50 years have had a sharp rising trend, the same cannot be said for unmarried women\footnote{Married women has a gone from 30\% to 60\% employment, where single and divorced women have been at around 70\% employment throughout the sample.}. The authors conclude that the rise in female employment in large part can be explained by the increase in years of schooling and the rise of female wages. They also find that changes in fertility do not have a big impact. 

Briefly discussing results in a Danish context. Using Danish data \textcite{kleven_children_2019} find that the long-term effects of the of a child reduces earnings by about 20 \%, and the hours worked by 10\% for women, while no effects are notable for men (10 year horizon). The same goes for participation rates which fall about 13 \% and wage rates which falls about 9 \% in the long-run for women (10 year horizon). \textcite{jorgensen_life-cycle_2017} investigates the relationship between consumption of non-durable goods and the average number of children, since these follow the same trajectory over the life cycle. He finds that income of the households fall, but in contrast with \textcite{kleven_children_2019} the economic effect is negligible with a reduction around 1\% in households with one child compared to households with no children. 

Considering that children can have an effect on the labour force participation of women one might ask why? This can be framed in two different ways. The first option is, that women want to spend more time with their kids because it is a more joyful activity. The second option is that some of the time children take up can be considered work, implying less leisure time for women, which causes women to work less. \textcite{firestone_estimation_1988} Investigates that latter hypothesis and find that  women will in general lose about 3 hours of leisure per week for each child. This falls in line with \textcite{thrane_men_2000} that finds that 0.31 hours per day of leisure is lost which on a weekly basis would accumulate to 2.2 hours a week. \textcite{ekert-jaffe_time_2015} finds that leisure time falls at about half an hour per day yielding 3.5 five hours of leisure foregone per week. However, for children below the age of three about 1.4 hours of foregone leisure time is observed, leading to about 10 hours of lost leisure time per week for women. These numbers corresponds to those found by \textcite{hotz_empirical_1988} which, as mentioned earlier, is about $12.7$ hours per week, falling geometrically. 