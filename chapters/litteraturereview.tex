\section{Review of Literature}\label{sec:lit_review}

There exists a deep literature on both labour supply and fertility.
As this paper tries to investigate the relationship between women's labour supply and fertility by formulating a dynamic  structural model, the focus will lie on literature that has the same scope. The paper that has been the main inspiration for this paper has been work by Francesconi \parencite{francesconi_joint_2002}, where he proposes a join dynamic model of fertility and labour supply of married women. Francesconi formulates a model with a joint decision of fertility and labour supply. Labour supply is considered in his formulation split into three choices: not work, work part-time and work full-time. Furthermore Francesconi proposes a model of human capital accumulation as a function of the labour supply. Additionally he assumes a budget constraint of all income in period $t$ should be consumed in period $t$, and the husbands income and labour supply is modelled to follow an exogenous process. 

Work has also been done in extending the basic framework proposed by Francesconi that models joint decision between fertility and labour supply.  The life cycle model has been extended by allowing for savings, and more sophisticated skill atrophy processes \parencite{adda_career_2011}, Allowing for the participation rate of women to be continuous \parencite{gayle_life-cyle_2006} and Keane and Wolpin focuses on the marriage market while still allowing for fertility and labour supply being part of the choice set \parencite{keane_role_2010}.

The joint decision between fertility and labour supply has been investigated before Francesconi investigated the problem by proposing a dynamic structural model \parencite{francesconi_joint_2002}. Both \parencite{moffitt_estimation_1984} and \parencite{hotz_empirical_1988} utilizes an empirical closed form approach uses a closed form solution to the problem of fertility and labour supply to estimate an econometric model. 