\section{Conclusion}

This paper has tried to establish how reinforcement learning can be used to solve dynamic models. I show that using these solution methods allow for solving models otherwise computationally infeasible, however when using such methods, you have no guarantee of finding the globally optimal policy, and there exists a real risk of ending in local maxima. Furthermore have I presented an economic model of exogenous fertility and endogenous female labour supply, that can give a surprisingly good fit to data, and show comparable long term penalties for female participation rates, earnings, wage rates and hours worked to contemporary findings by \textcite{kleven_children_2019}. 

The first part of the paper I draw on the literature of female labour supply and fertility, which is then used to formulate a simple dynamic model of female labour supply and fertility, tracking households over the life cycle, where women can supply a discreet number of hours to the labour force. Next, I present the reader for reinforcement learning and deep learning allowing the reader to have an understanding of the algorithms used for solving the models presented in this paper. Next I use aggregate data for Statistics Denmark to calibrate the parameters of the income process. For this task I assume that the households follows a deterministic strategy with regards to labour supply. I find, when using indirect inference, the model seem to approximate the income paths of both men and women very well. Next, I present three different solution methods: Value function Iteration using deep neural networks for value function approximation, Deep Q-learning and Double Deep Q-Learning. I find that these three solution methods have comparable performance, but the model does not seem to fit the data well! Next i formulate an extension of the simple model, and use Double Deep Q-learning as solution method. I estimate the extra parameters using method of simulated moments. Simulating from the model using the estimated parameters, I find that both the participation rate and the average number of supplied hours to the labour force is comparable to what is found in data. Next i compare event graphs to those found by \textcite{kleven_children_2019} and find striking similarities.