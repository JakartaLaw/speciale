\section{Introduction} 

The last 10 years have lead to numerous breakthroughs within the field of machine learning. The sub field reinforcement learning has been especially prolific. In 2013 the company DeepMind achieved super human performance in various Atari games \parencite{mnih_playing_2013}. In 2018 the same company beat professional human players in the board game Go, a feat which was not deemed possible in a foreseeable future \parencite{silver_general_2018}, and as late as 2019 DeepMind showed that a reinforcement learning agent was able to play the game Star Craft 2 on the same level as the best human players \parencite{vinyals_grandmaster_2019}. All the games mentioned can be considered dynamic models. Dynamic models play a central role within the field of economics. The results imply a new way of solving dynamic economic models using reinforcement learning.

Dynamical models usually involve agents that take sequential actions, trying to maximize the cumulative utility throughout the lifecycle. This class of economic models conform to a set of properties economists like: They have a micro foundation; agents are utility maximizing. They model time, allowing for agents to foresee the future, and act in accordance to their expectations. Some dynamic models can be solved  analytically. This is true for the canonical Ramsey model. However, when the scope of the model grows, different approaches are necessary. Dynamic programming is usually the tool utilized for solving such models.  Even though dynamic programming is a flexible tool, it does have its limitations. Solving a model using dynamic programming, requires a limited sized state space, otherwise the computation involved becomes infeasible. In practice this leaves high dimensional dynamic models impossible to solve using contemporary techniques. Deep reinforcement learning allows for solving such models. Because these techniques are relatively novel, they have not yet been introduced into the field of economics. Deep reinforcement learning, does unfortunately not guarantee that the solution converges to the global maximum, but results have shown that learning is possible in hard, high dimensional environments.

This paper uses the aforementioned techniques to investigate the effect of children on female labour supply. Inspired by the model specification of \textcite{francesconi_joint_2002} and \textcite{adda_career_2011} I formulate a discrete time, finite horizon model that models discreet female labour supply and its relationship to fertility. First a simple model is formulated, where women can choose the number of supplied hours, letting fertility be exogenous, with an income process following the Mincer equation of human capital. The husband of the household, is assumed to follow a deterministic path regarding the number of hours supplied to the labour force, and the wage rates they receive. Households are assumed to face a budget constraint, that neither allow for borrowing or saving. Utility is assumed to be a function of leisure and consumption, and children are assumed to reduce leisure by mirroring additional work for the woman, that is not financially compensated. Later an extension to the original model is presented with exogenous education combined with a transfer system for women in the education system. Additionally, the extension tracks children on an individual level.

Using three different solution methods (value function iteration, deep Q-learning and double deep Q-learning) I solve the simple model. I show that one can get comparable performance using deep reinforcement learning when compared to using value function iteration solution methods, yielding a new way to solve more complex dynamic models. The parameters of the Mincer equation are calibrated using data from Statistics Denmark. The model is estimated using method of simulated moments, where a simple grid search approach is applied due to fact the optimization problem being one dimensional. Only double deep Q-learning is used to solve the extended model. Again the model is estimated using method of simulated moments and grid search. The data used for the optimization is from Statistics Denmark.

My two main findings are: 1) Deep reinforcement learning can yield comparable performance to value function iteration solution methods. Considering this allows for solving dynamic models with high dimensional state space, I argue these methods should be explored further in the field of economics. 2) Simulating from the estimated model, I find the initial simple model is not able fit the data, whereas the extended model does fit the data surprisingly well. Both participation rates and average number of supplied hours to labour force, are surprisingly close to what the data from Statistics Denmark suggest. Comparing to  \textcite{kleven_children_2019}, this paper finds results very similar regarding earnings, participation rate, supplied hours to the labour force and wage rates, when a woman gives birth to a child.

The paper follows the structure: A literature review is conducted highlighting the main findings in articles of endogenous female labour supply and the effect of children. A model is formulated based on key takeaways from the literature. The parameters of the income process is calibrated using data from Statistics Denmark. Next, I introduce the reader to both reinforcement learning and deep learning. Settling on three different solution methods I solve and estimate the model. I go on to extend the model, and solve it using double deep Q-Learning. I end by comparing the results to contemporary findings, and data from Statistics Denmark.